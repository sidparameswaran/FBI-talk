\begin{frame}{Free Fermion Entanglement}

\only<1>{
Entanglement in free fermions:

Haldane Haldane Haldane
}

%Classification of quadratic free fermion hamiltonians
%involves fairly sophisticated math
%I just want to point out the following features of a classification like this: there are a number of Z or Z2 topological invariants, each of which that can only change under phase transitions
%These invariants aren't always very physical, and while they distinguish free fermion Hamiltonians, they aren't guaranteed to stick around when you add interactions.
% Most of the invariants are only well defined in the precense of some additional symmetry
% There is an even bigger table when you include lattice symmetries (Topological Crystalline Insulators)
\only<2>{
\begin{figure}
\includegraphics[width=\linewidth]{diagrams/AZ.png)
\caption{Free-fermion classification (ten-fold way) for $\mathcal{T}, \mathcal{C}, \mathcal{TC}$} 
\end{figure}
\footnotetext[5]{
\citep{Ryu2009-qc}} 
}
\only<3>{
Mathematical classification:
\bi 
\item A number of {\em topological} invariants, which are in this case either integers or $Z_2$-type invariants
\bi
\item Discrete and continuous under adiabatic changes of $H$
\ei
\item Some only well defined in the presence of some symmetry
\item Some can only be non-zero in the absence of some symmetry
\item Invariants may not be physical - not guaranteed to make sense in interacting problems
%In particular, the invariants for Topological Crystalline Insulators depend explicitly on solving for band wave functions
\ei 
}



\end{frame}