\begin{frame}
  \only<1>{
  \frametitle{What is a tensor network?}}
  \only<2->{
  \frametitle{What is a tensor network {\em state}?}}
  \bi
  \item ... a contraction scheme for building tensors
  \begin{figure}[t]
                \begin{tikzpicture}[node distance=0.4cm]

\node[det] (t) at (-1,-3.8) {$A$};
\node[det] (q) at (1,-3.8) {$B$};
\node (ti) at (-2,-3.8) {$i$};
\node (qn) at (2,-3.4) {$n$};
\node (qr) at (2,-4.2) {$r$};
\draw[thick] (ti) -- (t);
\draw[thick] (q) -- (qn);
\draw[thick] (q) -- (qr);
\draw[thick] (t) to [bend right=30] node[below] {$k$} (q);
\draw[thick] (t) to [bend left=30] node[above] {$j$} (q);
%\node[below of=q] {$q$};
%\node[below of=t] {$t$};
\node at (5,-3.8) {$\sum_{jk} A_{ijk} B_{jknr}$};

\end{tikzpicture}
    \end{figure}
  \pause
  \item ... an ansatz for wavefunctions using a tensor network
  \begin{figure}[t]
                \begin{tikzpicture}[node distance=0.4cm]
\tikzset{gamma/.style={circle=2pt,draw=black!100, very thick, fill=blue!40, inner sep=3pt}}


\begin{scope}[decoration={
    markings,
    mark=at position 0.5 with {\arrow{>}}}
    ]

\foreach \i in {1,...,6} {
	\node[gamma] (G\i) at (1.5*\i,0) {$A$};
	\node (l\i) at ($ (G\i) + (0, 1) $) {$$};
	\draw[very thick, red!100, postaction={decorate}] (G\i) -- (l\i);
	%\node[below of=G\i] {$A$};
}

\foreach \i in {7} {
	\node[] (G\i) at (1.5*\i,0) {$...$};
    }  
    
\foreach \i / \j in {1/2,2/3,3/4,4/5,5/6, 6/7} {
	\draw[very thick, postaction={decorate}] (G\j) -- (G\i);
}
\node (b0) at (0, 0) {$...$};
\draw[very thick, postaction={decorate}] (G1) -- (b0);

\end{scope}
\end{tikzpicture}

        \caption{Matrix Product State ansatz}
    \end{figure}
    \note{contraction scheme is one tensor per site}
   \note{Reason for title matrix product state: each w.f. coefficient is computed
         using a product of matrices}
   \note{Goal is to find a local computational procedure for specifying coefficients of         the wavefunction}
   \note{Similar but not quite equivalent to specifying a circuit of local quantum gates          to construct the state}
  \note{number of parameters linear in system size, fixed bond dimension}
  \note{bond dimension may need to go up to improve wavefunction accuracy}
  \note{given such a procedure, we automatically satisfy entanglement bounds}
   \note{In order for this to not be every wavefunction, have to fix bond dimension}
  \ei
\end{frame}