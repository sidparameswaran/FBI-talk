\section{Symmetry protection}
\label{sec:symmetry}

Notice that many points in the entanglement spectrum are doubly degenerate - those that are
assigned a non-zero $U(1)$ charge - and on odd circumference cylinders, the
entire entanglement spectrum is doubly degenerate. In this section, we will discuss how
the corresponding degenerate Schmidt states are related by a symmetry of the HFBI wavefunction. 
As discussed in Ref.~\onlinecite{pollmann2010} and reviewed in the Appendix~\ref{Appendix:MPS},
this symmetry action can be used to diagnose one-dimensional symmetry protected topological order.
for which degeneracy throughout the entire entanglement spectrum is a robust feature.
We will demonstrate that the odd circumference cylinders, considered as one-dimensional states, 
are indeed SPTs protected by a combination of lattice inversion and charge parity symmetries.

Unlike the entanglement spectrum, the choice of Schmidt states is not unique when the spectrum is 
degenerate. A change of basis $V^{ji}$ that mixes degenerate Schmidt states for the left half 
cylinder produces an equally good Schmidt decomposition, as long as the right half cylinder states 
are transformed using the complex conjugate basis change $V^{ji*}$. It is known that any two
Schmidt decompositions must be related by such a $V$.

%If 
%
%\beq
%\label{eq:schmidtredundancy}
%\ket{\widetilde{\psi^{(i)}_{L}}} = \sum\limits_j \ket{\psi^{(j)}_{L}} V^{ji} \text{ and }
%\ket{\widetilde{\psi^{(i)}_{R}}} = \sum\limits_j \ket{\psi^{(j)}_{R}} V^{ji*},
%\eeq
%
%then 
%
%\begin{equation*}
%\ket{\psi} = \sum\limits_i \sqrt{\rho_i}
%\ket{\widetilde{\psi^{(i)}_{L}}}\ket{\widetilde{\psi^{(i)}_{R}}}.
%\end{equation*}

Since an on-site symmetry $U_g$ (or more generally, any symmetry which acts
separately on the left and right half of the cylinders) can be used to produce another Schmidt
decomposition, the action of $U_g$ on the Schmidt states must take the form
\beq
\label{eq:symschmidt}
\begin{split}
U_g \ket{\psi^{(i)}_{L}} &= \sum\limits_j \ket{\psi^{(j)}_{L}} V_g^{ji} \\
U_g \ket{\psi^{(i)}_{R}} &= \sum\limits_k \ket{\psi^{(k)}_{R}} V_g^{ki*},
\end{split}
\eeq
where the $V_g^{ji}$ are unitary matrices that only act on degenerate blocks of Schmidt states.
The $V_g$ can be computed numerically for each on-site symmetry $g$.

We can also analyze the effects of symmetry that preserve the entanglement cut line but swap
the left and right of the cylinders. In general, we will consider 
any symmetry $h$ that swaps the cylinder halves and square to the identity, which we will call an 
inverting symmetry. These satisfy a modification of equation~\ref{eq:symschmidt}:
\beq
\label{eq:isymschmidt}
\begin{split}
U_{h} \ket{\psi^{(i)}_{L}} &= \sum\limits_j \ket{\psi^{(j)}_{R}} V_{h}^{ji} \\
U_{h} \ket{\psi^{(i)}_{R}} &= \sum\limits_k \ket{\psi^{(k)}_{L}} V_{h}^{ki*}.
\end{split}
\eeq

As reviewed in Appendix~\ref{Appendix:MPS}, the collection of $V_g$ of on-site symmetries 
sometimes fail to satisfy the group multiplication laws on the nose, but instead form a 
projective representation, characterized by phases $\omega(g_1, g_2)$ that measure the failure to 
satisfy the group laws. 
Additionally, the $V_h$ for each inverting symmetry forms an anti-unitary projective 
representation of $\mathbb{Z}_2$ - or in other words, they satisfy
$$
(V_h \mathbf{K})^2 = V_h V_h^* =  e^{i \phi_h} I = \pm I,
$$
where $\mathbf{K}$ represents complex conjugation in the canonical basis. 

The phases $\omega(g_1, g_2)$ can be assigned to one of a discrete class of projective 
representations labeled by an element of $H^2(G, U(1))$. 
If the element is non-trivial, the degeneracy cannot be removed without breaking the symmetry, 
since the distinct classes of $H^2(G, U(1))$ cannot be connected continuously.
Similarly, for the inverting symmetries $h$, the phase $\phi_h = -1$ signifies that 
the degeneracy cannot be removed without breaking the symmetry.

The on-site symmetries of this wavefunction are the $U(1)$ charge symmetry and the 
anti-unitary symmetry $\tau$, which acts by complex conjugation in the boson number basis.
Despite the half-filling, a particle-hole transformation does not preserve the hard-core 
variant of the HFBI, so this is not an additional symmetry of the wavefunction. We find
that the edge action of this symmetry group does not form a projective representation,
and so does not protect the odd $W$ cylinder degeneracy. In order to protect the degeneracy, we must expand the group of interest to include lattice symmetries.

In the cylinder geometry chosen, the group of lattice symmetries consists of translations 
- generated by $T_x$ parallel and $T_y$ perpendicular to the cylinder axis-
as well as reflections $\I_x$ through a line parallel 
and $\I_y$ through a line perpendicular to the cylinder axis. 
We also consider lattice inversion $\I = \I_x \I_y$, equivalent to a $\pi$ 
rotation of the spatial plane about the center of a hexagonal plaquette.

We can immediately see from symmetry considerations that $V_{\I}$ will map each Schmidt 
state to a state with the same entanglement energy and momentum, but with opposite charge.
\brayden{explanation}
Thus $V_{\I}$ acts on the degenerate pairs of states $\ket{e, K}$ and $\ket{-e, K}$ which
appear in Figure~\ref{fig:ESL910}. 

Our numerical results show that 
$$
V_{\mathcal{I}} V_{\mathcal{I}}^* = I, 
$$
but
$$
V_{\varPi \mathcal{I}} V_{\varPi \mathcal{I}}^* = -I 
$$
where $\varPi = e^{i \pi \mathcal{Q}} \in U(1)$ is the charge parity symmetry.


%However, the entanglement spectrum of this one wavefunction is not enough to
%determine the SPT nature of the phase. First, it isn't clear what group or
%groups of symmetries can be used to protect the gapless edge. It could be that
%the entire symmetry group - $U(1)$ boson number conservation, \brayden{time
%reversal symmetry?}, as well as the entire rotational and translational group
%of the lattice - must be preserved to protect the edge. Or, as we will argue
%is the case, a much smaller subgroup could be used to protect the edge. This
%leads to a wider class of perturbations that leave the edge intact - and since
%we will argue that translation is not needed for protection, the gapless edge should additonally be robust to some types of weak disorder and can be seen in
%small systems with symmetry preserving boundary conditions. Second, the
%entanglement spectrum alone fails to distinguish between other SPT phases with
%the same protecting group - for that, we need a topological invariant. Third,
%we would like to confirm the SPT nature of the phase by perturbing a parent
%Hamiltonian with terms that break various combinations of symmetries and
%seeing if they destroy the gapless edge.

%To determine the protecting group, we can proceed in
%two ways: we could perturb a parent Hamiltonian with terms that break various
%combinations of symmetries and see which perturbations destroy the gapless
%edge, or we can look for a topological invariant by analyzing the action of
%the symmetry of the entanglement edge. We will take up the later question
%first, then return to the prospect of perturbing a parent Hamiltonian in
%Section \ref{sec:perturbations}.


%Given that only odd circumference cylinders are SPTs, one might wonder whether
%odd and even $L$ cylinders approach two differt phases in the thermodynamic
%limit. In Section~\ref{sec:CFT}, we will provide evidence against that
%possibility, by showing that in both cases, the low-energy, linear dispersing
%part of the entanglement spectra can be described by the same conformal field
%theory. Thus, in the thermodynamic limit, the two edge spectra approach the
%same set of points.



\brayden{1D SPT -> 2D SPT+translation:
The degenerate entanglement spectra on odd circumference cylinders should be a feature 
robust to small perturbations of any parent Hamiltonian for this state, as long as the 
perturbations respect the above symmetry, and as long as perturbations respect the translational
symmetry of the lattice - so that the notion of 'odd circumference cylinder' is well defined.
So the 1D SPT result can be extended to a 2D SPT result, as long as we include translational
symmetry in the protecting group. It may be that translational symmetry isn't actually necessary
to argue for this protection, but additional evidence is needed to argue for that. 
}