%!TEX root = ../fbi.tex

\section{Conclusion and Discussion}

We have applied recently developed tensor network methods to study the edge properties
of a bosonic insulator that is featureless in the bulk. Our simulations are performed for an infinitely
long cylinder of finite circumference $W$. This allows us to numerically extract the exact entanglement
spectrum for up to $W = 10$. We find that the entanglement gap closes as $1/W$, and that furthermore
the low-lying spectrum coincides to high accuracy with the spectrum of a free boson conformal field theory.
This is further corroborated by observing a central charge of $c=1$ in the entropy of the lowest Schmidt state.

While these observations are consistent with and strongly suggestive of a symmetry-protected topological
phase, where such a gapless spectrum would naturally emerge at the edge, our calculations do not establish
a strong connection between the edge spectrum and symmetry-protection, i.e. they leave open the possibility
that the gapless entanglement spectrum is accidental.
To make progress on this question, we analyze in some detail the exact degeneracies in the entanglement
spectrum for cylinders of odd circumference $W$. Using recently developed tools based on matrix-product
states, we are able to establish a strong connection to the symmetries of the state by computing topological
invariants that detect the non-trivial action of certain combinations of lattice and spin symmetries on the edge.
This establishes in the affirmative that the quasi-one-dimensional systems obtained for odd $W$ represent
one-dimensional SPT phases.

We cannot establish with the same reliability that the symmetries that protect these one-dimensional topological
invariants also protect the gapless edge spectrum on the edge of the two-dimensional system. However, several
considerations are in favor of this.
Firstly, we observe that the symmetries that are shown to be relevant to the case of odd $W$ are not
inherently one-dimensional and could apply equally well to the full, two-dimensional system.
Furthermore, we can formulate arguments based on the structure of the tensor network representation.
As outlined in Section~\ref{sec:ES}, the edge of the tensor network representation for the cut chosen here
can be represented in a Hilbert space of hard-core bosons hopping on a one-dimensional chain with one site
per plaquette, where the occupation of a site corresponds to whether the boson of that plaquette is found
on the left or right side of the cut. In this representation, the reflection symmetry about the cut takes
the special role of guaranteeing equal probability for the boson to be on the left or right of the cut, and thus fixing
the model for the edge to half-filling. If translational symmetry is enforced, a Lieb-Schultz-Matthis
theorem applies and guarantees that the entanglement edge is either gapless or spontaneously breaks a
symmetry, which would both be consistent with a symmetry-protected topological phase.



\acknowledgements
The authors would like to thank A. Vishwanath for useful discussions.
B.W. would like to acknowledge Eugeniu Plamadeala for helpful discussions.
