%!TEX root = ../fbi.tex

\section{Correlation functions}

In Ref.~\onlinecite{kimchi2013}, certain real-space correlation functions
of the featureless boson insulator state were studied using a mapping to
a particular classical statistical mechanics system which was sampled using
Monte Carlo techniques.
Here, we employ PEPS calculations on an infinite cylinder and first compare
against the known results from variational Monte Carlo and then go beyond
these established results by measuring additional off-diagonal correlation
functions and establishing a strict upper bound on the exponential decay
of \emph{all} two-point correlation functions for an infinite cylinder of
given width.

It is a well-known result that PEPS can, in the thermodynamic limit,
exhibit power-law correlation functions~\cite{?}, while the correlation
functions in an MPS of finite bond dimension are necessarily either constant
(or oscillating with constant amplitude) or exponentially decaying. The
long-range correlations of an MPS are encoded in its transfer operator $T$,
which for an MPS of bond dimension $M$ is a matrix of size $M^2 \times M^2$.
Denoting the spectrum of $T$ as $\lambda_i$, $|\lambda_0| \geq |\lambda_1| > \ldots$,
we can normalize the state such that $\lambda_0 = 1$. If the largest eigenvalue
is found to be non-degenerate, $\lambda_1 < \lambda_0$, we have that all correlation
functions of operators $\mathcal{O}_i$ that are supported on a finite number of sites centered
around a site $i$ decay as
$\langle \mathcal{O}_i \mathcal{O}_j \rangle - \langle \mathcal{O}_i \rangle \langle \mathcal{O}_j \rangle \sim e^{-|i-j|/\xi_\mathcal{O}}$.
Crucially, the correlation length $\xi_\mathcal{O}$ for any operator $\mathcal{O}$
is bounded from above by $-\log(\lambda_1)$~\cite{?}. In the following, we thus
evaluate the spectrum of the transfer operator of our PEPS along cylinders of varying
circumference $L$ to establish an upper bound on the correlation length for each
circumference $\xi(L)$.
Note that the possibility of having power-law correlations in a PEPS can
be reconciled with the above consideration if the correlation length $\xi(L)$
diverges as $L \rightarrow \infty$; we will thus need to carefully consider the scaling
of $\xi(L)$.

\bela{Now a paragraph describing the comparison to Monte Carlo results.}

Our results for the correlation bounds $\xi(L)$ are shown in Fig.~\ref{fig:TMS}.
Here, we show the upper bound for the case of soft-core and hard-core bosons,
and in each case consider the spectrum of the full transfer operator as well
as the transfer operator restricted to the sector that is diagonal in particle
number and thus encodes correlations of operators $\mathcal{O}_i$ that do not
change the boson number (such as density-density correlations).
We find that in each case, the largest eigenvalue of the transfer operator is
non-degenerate. Furthermore, we find
that the correlation length approaches a finite constant as we
increase $L$, as shown in Figure \ref{fig:TMS}.

This 1D correlation length bounds all correlations stretching along
the cylinder, but it is possible that placing the state on a cylinder
- which explicitly breaks the rotational symmetry of the honeycomb
lattice - leads to anisotropic correlations, even in the large $L$
limit. By looking at explicit short-distance correlation functions
$\ev{b_i^{\dagger} b_j}$ and $\ev{n_i n_j}$, we see that instead the
rotational symmetry is restored in the large $L$ limit. Our maximum
cylinder circumference $L=10$ is several times larger than the
measured correlation length $\xi \approx 3.2$.

\begin{figure}[hbtc]
	\centering
	\includegraphics[width=\columnwidth]{TransferMatrixSpectrum_big.pdf}
	\caption{Placeholder for plots showing correlation bounds.}
	\label{fig:TMS}
\end{figure}

\begin{figure}[hbtc]
	\centering
	\includegraphics[width=\columnwidth]{ShortDistanceCorrelations.pdf}
	\vskip-3cm %
	\caption{\brayden{Replace test data function with real data} Short distance correlation functions $\ev{b_i^{\dagger} b_j}$ (on the left) and $\ev{(n_i-\frac12)(n_j-\frac12)}$ (on the right). }
	\label{fig:ShortCorr}
\end{figure}
