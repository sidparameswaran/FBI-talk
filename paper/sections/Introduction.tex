%!TEX root = ../fbi.tex

\section{Introduction}

The discovery in 2007 of time-reversal invariant band insulators that are not adiabatically connected to the atomic limit~\cite{...}
has spurred the exploration of a broad array of phases where symmetries protect subtle, non-local features that distinguish
them from trivial, unentangled insulators. These phases, colectively known as symmetry-protected topological (SPT) phases~\cite{...}, have
by now been observed in several (so far, weakly-interacting) experimental realizations both in two and three dimensions~\cite{somereview} and an extensive
mathematical framework has been developed for their characterization and classification~\cite{wen...}.

While the bulk of the system in such a phase is often only subtly distinguished from a trivial insulator, the non-local or topological properties can be observed by focusing on the boundary of a finite system.  The boundary often exhibits gapless features such as the localized
Majorana zero modes at the ends of one-dimensional topological superconductors~\cite{kitaev2001},
the helical edge states of the quantum spin Hall effect~\cite{...} and the protected Dirac cones on the surface of three-dimensional
$\mathbb{Z}_2$ topological insulators~\cite{...}.
It has been shown that these features also appear in the entanglement spectrum~\cite{li2008}, where they generally take the form
of either protected degeneracies or gapless spectra that mirror the gapless modes on a boundary.
These signify entanglement that cannot be removed while preserving the symmetry, and can thus be used to establish
that SPT phases are fundamentally distinct from trivial, non-entangled insulators.

Symmetries which relate the physical locations of degrees of freedom, such as the spatial symmetries of rotation or reflection, are often observed to be fundamentaly distinct from on-site symmetries (such as charge conservation or time-reversal symmetry) in this context.  In particular, the interplay of on-site and spatial symmetries can protect new kinds of topological phases.  \cite{...}
% cite: liang fu Topological crystalline insulators, liang fu and exptalists TCI discovery,  turner etc paper on haldane chain prtected also by inversion
The role of the boundary is also modified when spatial symmetries are considered; when the symmetry may be broken by a physical boundary of a system, as for instance occurs for the case of inversion symmetry on any boundary, the non-trivial features can still be extracted from the entanglement spectrum.  The combination of entanglement and physical gapless modes thus permits topological phases to be theoretically distinguished from trivial insulators in a transparent way, even when a necessary ingredient for the topological protection is a non-local symmetry operation. 
However, while non-interacting crystalline topological insulators, whose non-trivial topology requires some non-local symmetry operations, have been well explored~\cite{...}, far less is known about their interacting counterparts in two or three dimensions. 

In this work, we compute the entanglement propeties of an insulating state of interacting bosons on the honeycomb lattice, and show that it constitutes a topological phase protected by lattice symmetries.  In particular, we show that the non-trivial entanglement is intimately related to the representation of the lattice symmetries as the honeycomb (or graphene) tight-binding model. We numerically represent the quantum state as a network of tensors, arranged to form the honeycomb lattice tight-binding model. We find that the action of charge conservation and spatial symmetries on certain entanglement cuts combines to enforce entanglement degeneracies, which remain protected as long as the tight-binding lattice is preserved.

The quantum wavefunction in the present study, first proposed in Ref.~\onlinecite{kimchi2013},
is a type of Mott insulating state which requires a non-Bravais lattice, i.e. a lattice with a non-trivial unit cell. 
The necessity for a unit cell arises due to the Lieb-Schultz-Matthis (LSM) theorem~\cite{...}.
The LSM theorem forbids the existence of a featureless state
-- a state that neither spontaneously breaks a symmetry, nor displays topological
order, nor has power-law correlations and is thus ''gapless'' -- in systems
with a fractional filling per unit cell. In contrast, the classical cartoon of a Mott insulator requires a filling which is not only integer per unit cell but also integer per site. A featureless insulating state is in principle allowed by the LSM theorem on the honeycomb lattice
at half filling per site, and thus unit filling per unit cell. Such a state is necesarily distinct from the usual integer-per-site Mott insulator. 
Moreover, despite being a state of particles at half filling on a lattice with an even number of sites per unit cell, this state is in no way related to a non-interacting band insulator.  Indeed, here no band insulator can exist: symmetry guarantees that a free-fermion spectrum is gapless at certain high-symmetry points, and there are
thus no localized Wannier basis.
This implies that a featureless state on the honeycomb lattice cannot be constructed by filling a permanent of localized Wannier
orbitals~\cite{parameswaran2013}, and any construction of a quantum state must thus involve interactions in a non-trivial way. 
Ref.~\onlinecite{kimchi2013} pursued
an approach of constructing a permanent wavefunction by filling local and symmetric
orbitals that are not orthogonal and
it was argued that that the resulting wavefunction is indeed featureless.
In particular, using numerical simulations it was found that the state exhibits
isotropic and exponentially decaying correlations, and arguments were presented that it is not
topologically ordered.

While we confirm the featureless bulk of the state, we show that nevertheless the entanglement of the state is non-trivial cannot be removed while preserving
all symmetries, i.e. it constitutes a symmetry-protected phase. The relevant symmetry is a combination of charge
conservation and lattice symmetries, which together protect universal features in the entanglement spectrum. In
particular, we show that the low-lying entanglement spectrum is to great accuracy described by that of a $c=1$
conformal field theory, and that there is an exact double 
degeneracy throughout the entanglement spectrum for certain geometries, which is protected
by the symmetries of the state and thus serves as a topological invariant identifying the SPT order.
Since lattice symmetry is involved crucially, this provides one of the first examples for an SPT of interacting bosons
protected by lattice rather than on-site symmetries.

All of these properties of the phase become accessible through a description of the state
as a projected entangled-pair state (PEPS)~\cite{verstraete2004}. These states form a specific class of tensor
network states that corresponds to a generalization of the well-known matrix-product state
(MPS)~\cite{white1992,ostlund1996,schollwoeck2010}
framework to higher dimensions. PEPS have been shown to be a powerful description of many
classes of gapped systems, including topologically ordered and SPT phases. Here, we have an
exact description of the state as a PEPS, allowing us to extract properties such as the entanglement
spectrum and the topological invariants exactly on certain geometries; we emphasize that these
properties of the state are not accessible to other numerical methods.

The topological invariants extracted here form examples of a broad class
of invariants that provably must be constant throughout the phase. These differ
from the order parameters that measure local symmetry breaking in that they
are not related to the expectation values of local operators. Early examples
of topological invariants for SPT phases are the string order parameter for the one-dimensional
AKLT phase~\cite{...}, and the spin Chern number for the quantum spin Hall effect~\cite{...}.
The invariants we consider here measure how the action of the symmetry is implemented on the 
physical edge states of open systems or on the Schmidt states of an 
entanglement decomposition~\cite{pollmann,...}. These invariants feature heavily in the 
classification of SPT phases with on-site symmetry~\cite{...}, and similar invariants 
that apply to free-fermion states have been used for topological crystalline 
insulators~\cite{bernevig...}. By contrast, topological invariants for interacting states 
protected by lattice symmetries in more than one dimension are not well 
understood. We will discuss the action of the symmetry on the edge of the  
state and progress towards the goal of finding a topological invariant to 
identify the corresponding phase.
