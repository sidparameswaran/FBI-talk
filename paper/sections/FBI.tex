%!TEX root = ../fbi.tex

\section{Construction of the featureless boson insulator}
\label{sec:fbi}

In the honeycomb lattice, each unit cell is associated with exactly one hexagon plaquette, which respects the lattice point group symmetries. As shown in Ref.~\onlinecite{kimchi2013}, this provides an explicit
construction of a bosonic insulator
on the honeycomb lattice that is completely featureless in the bulk,
henceforth referred to as \emph{honeycomb featureless boson insulator} (HFBI).
The state is succinctly described by the following wavefunction expression:
\begin{equation} \label{eq:def}
\ket{\psi} = \prod\limits_{\varhexagon} \left( \sum\limits_{i \in \varhexagon} b^{\dagger}_{i} \right) \ket{0}.
\end{equation}
Here, $\varhexagon$ denotes the elementary hexagons of the honeycomb
lattice. Despite the compact notation, this many-body bosonic state involves many overlaps of multiple particles and requires concrete computation for its properties to be unveiled.

We focus on two closely related variants of this state: a
version of soft-core bosons where $b_i^\dagger$ creates a boson on
site $i$ and obeys the usual bosonic commutation relations, and a
hard-core version of the same state where $b_i^\dagger$ also creates a
boson but $(b_i^\dagger)^2=0$. In either case, the operator $\sum_{i
\in \varhexagon} b^{\dagger}_{i}$ creates exactly one boson per
hexagon; as there is one hexagon per unit cell of two sites of the
lattice, the state has one boson per unit cell, or half a boson per
site, thus allowing the existence of a featureless state.
In the case of soft-core bosons, the maximum number of bosons
per site is three.

\bela{Summarize what was calculated in Ref.~\onlinecite{kimchi2013}.}

\subsection{PEPS representation}

In order to make the state~\eqnref{eq:def} more amenable to numerical
simulations, and in particular in order to be able to study its edge
properties, we now derive a representation as a projected entangled
pair state (PEPS). Importantly, this PEPS description will respect
all of the relevant symmetries of \eqnref{eq:def}.

\begin{figure}
	\centering
	\includegraphics[width=0.8\columnwidth]{Hex_PEPS.pdf}
	\caption{Honeycomb lattice PEPS and zig-zag entanglement cut.
	%A PEPS on the honeycomb lattice in the "zig-zag"' cylinder geometry used
	%througout in this paper. 
	%The cylinder is created by periodically identifying sites $W$ unit cells away 
	%- in the picture, the 
	In this PEPS of rank-4 tensors, the top and bottom edges are identified, forming a cylinder with circumference $W=3$ unit cells. 	
	%The tensors A and B are rank 4, with one physical leg and 3 virtual legs at each site. 
	A one-dimensional MPS representation is constructed by blocking the tensors on each cylinder slice (green dotted lines).
	The entanglement cut shown (bold dotted line) passes through the hexagon mid-point, preventing the tight-binding lattice from gaining additional sites as long as crystalline symmetries are preserved. 
	%creates a MPS representation for the state.
	}
	\label{fig:PEPS}
\end{figure}

To obtain a PEPS construction, we first choose a local basis $\ket{n}$
of boson occupation numbers, i.e. $b^\dagger b \ket{n} = n \ket{n}$.
The PEPS will thus describe the coefficients of $\ket{\psi}$ in this
basis, $\langle n_1 \ldots n_L | \psi \rangle$. The PEPS
representation is most easily obtained in a two-step construction,
where we first construct the state shown in Fig.~\ref{fig:FBI_PEPS}.
Here, the tensor labeled $W=W^{n_1 \ldots n_6}$, which is placed in
the center of each hexagon, is a rank-6 tensor given by
\begin{equation} \label{eq:W}
W^{n_1 \ldots n_6}  = \left\{ \begin{array}{lr}
													1  : & \sum\limits_i n_i = 1 \\
													0  : & \text{else}
													\end{array} \right.,
\end{equation}
where each $n_i \in \{0, 1\}$.

\begin{figure}
	\centering
	\includegraphics[width=0.8\columnwidth]{FBI_PEPS.pdf}
	\caption{
	Intermediate tensor network for HFBI state. Here, the tensors labeled
	$D$ are located on the sites of the honeycomb lattice, while the
	tensors labeled $W$ are located on the centers of each hexagon.
	Dotted lines thus represent the physical lattice, while the solid
	lines indicate auxiliary bonds over which the tensor network is
	contracted.}
	\label{fig:FBI_PEPS}
\end{figure}

This tensor describes the coefficients of a so-called $W$-state in the
occupation number basis, i.e. $W^{n_1 \ldots n_6}= \langle n_1
\ldots n_6 | \sum_{i=1}^6 b_i^\dagger |0\rangle$. We note that this
tensor is symmetric under permutations of its indices.

On the sites of the physical lattice, we have placed a rank-4 tensor
denoted as $D$, shown in panel (a) of Fig.~\ref{fig:FBI_PEPS_2}, which
connects the $W$ tensors from three adjacent hexagons, and as fourth
index has a physical index $p$. For a state of soft-core bosons, where
$p=0,1,2,3$, this tensor is given by
\begin{equation} \label{eqn:Dsc}
D^\mathrm{sc}_{p, i_0 i_1 i_2}  = \left\{ \begin{array}{ll}
													\sqrt{p!}  &: p =i_0+i_1+i_2  \\
													0  &:  \text{else}
													\end{array}
											\right. .
\end{equation}
We can also encode a state of hard-core bosons by replacing $D$ by
\begin{equation} \label{eqn:Dhc}
D^\mathrm{hc}_{p, i_0 i_1 i_2}  = \left\{ \begin{array}{ll}
													1  &: p = i_0+i_1+i_2 \le 1  \\
													0  &:  \text{else}
													\end{array}
											\right.
\end{equation}
Other values for the $D$ and $W$ tensors that respect the charge and lattice symmetries
can also give rise to featureless insulators. 
Some of these variants are described in Appendix~\ref{Appendix:Variants}.

This tensor network wavefunction manifestly respects all the
translational and point group symmetries of the honeycomb lattice,
since the tensors $W$ and $D$ are invariant under rotations of their
virtual indices in the plane. One can also check that the
wavefunction is $U(1)$ invariant with charge $1$ per plaquette.

\begin{figure}
	\centering
	\subfigure[D tensor]{%
		\includegraphics[width=0.18\columnwidth]{D_op.pdf}
		\label{fig:D}
	}
	\quad
	\subfigure[W-tensor and factored form]{%
		\includegraphics[width=0.5\columnwidth]{w_string.pdf}
		\label{fig:W}
	}
	\subfigure[PEPS tensor network for F.B.I. state]{%
		\includegraphics[width=0.8\columnwidth]{FBI_PEPS_2.pdf}
		\label{fig:FBI_PEPS_2}
	}
\label{fig:PEPSforFBI}
\caption{Construction of PEPS for HFBI state. 
The site tensors (shown in the shaded regions of panel (c)) are constructed
using the factors of the plaquette tensor $W$ (panel (b))
combined with the original vertex tensor $D$ (panel (a)).
The red line in panel (c) shows where the entanglement cut considered in this paper
crosses the network.
}
\end{figure}

In order to convert the tensor network of Fig.~\ref{fig:FBI_PEPS} into a PEPS representation,
we first factor the $W$-tensor into a matrix-product state
of six tensors as shown in panel (b) of Fig.~\ref{fig:PEPSforFBI}. We
choose a form of the MPS that breaks the rotational symmetry
of the W-state (which appears as translational symmetry of the MPS).
This allows us to obtain an MPS description with a small bond dimension
of $M=2$; a fully symmetric choice would require bond dimension 6.
Since these states are physically equivalent, we expect all
physical quantities to be unaffected by this choice.
One possible decomposition is given by
\begin{equation}
W^{i_1 i_2 i_3 i_4 i_5 i_6} = \sum\limits_{\alpha_1 \ldots \alpha_5} V^{i_1}_{\alpha_1} W^{i_2}_{\alpha_1 \alpha_2}
\ldots
W^{i_5}_{\alpha_4 \alpha_5} X^{i_6}_{\alpha_5}
\end{equation}
where $V^{i_1}_{\alpha_1} = \delta_{i_1, \alpha_1}$, $X^{i_6}_{\alpha_5} = \delta_{i_6,\alpha_5+1}+\delta_{i_6,\alpha_5-1}$, and
\begin{equation*}
W_{i_0 i_1}^{j}  = \left\{ \begin{array}{ll}
													1  &:  i_0+j=i_1 \\
													0  &:  \text{else}
													\end{array}
											\right.,
\end{equation*}
where each index takes values in $\{0, 1\}$. Applying this to each
$W$-tensor yields the state as shown in panel (c) of
Fig.~\ref{fig:PEPSforFBI}. By contracting the four tensors in each
shaded region together, we obtain a PEPS in the regular form as shown
in Fig.~\ref{fig:PEPS}. The resulting PEPS has a bond dimension of
$M=2$ on the horizontal bonds, and a bond dimension of $M=4$ on all
other bonds. While it superficially breaks
the rotational symmetry of the lattice, it is an exact representation
of the FBI state and does not break any symmetries after contracting
the indices.

This decomposition respects the physical U(1) charge conservation
symmetry in that all tensors are separately
U(1)-invariant~\cite{bauer2011}. To make this manifest, we have
indicated in Fig.~\ref{fig:PEPSforFBI} arrows on each bond that show
the flow of charge.

\subsection{Representation on infinite cylinders}
For the calculations presented in this manuscript, we consider the
state $\ket{\psi}$ on a cylinder of infinite length, but finite
circumference $W$. In Fig.~\ref{fig:PEPS}, we have indicated the
choice of boundary conditions for the cylinder used in this paper. 
For many practical purposes, the PEPS on an infinite cylinder can be represented as an infinite, \
translationally invariant matrix-product state of bond dimension $\chi=2^W$
and physical dimension $p=4^{2W}$ ($p=2^{2W}$) for the soft-core (hard-core) case. 
The MPS is created by blocking all tensors in each slice of the cylinder, as shown in 
Figure~\ref{fig:PEPS}.

With each cylinder slice blocked together and considered as an MPS,
the procedures we use for computing both correlation functions and entanglement properties
are in principle identical to those used previously in MPS \cite{cirac2011,pollmann2010}.
Due to the exponential increase in the MPS bond dimension, this numerically exact approach
scales exponentially in the circumference of the cylinder.
It is however computationally advantageous to exploit the additional 
structure present in the PEPS transfer operator; 
by doing so, we can compute correlation functions and
the entanglement spectrum for the cut shown in Figure \ref{fig:FBI_PEPS_2}
for the HFBI state on cylinders of circumference up to $W=10$.
These computations are presented in the following 
sections~\ref{sec:Correlations} and \ref{sec:ES}.
