\documentclass[twocolumn,english,prb,showpacs,superscriptaddress]{revtex4-1}
\usepackage[colorlinks=true,urlcolor=blue,citecolor=blue,linkcolor=blue]{hyperref}
\usepackage[T1]{fontenc}
\usepackage[latin9]{inputenc}
\usepackage{amssymb}
\usepackage{graphicx}
\usepackage{amsmath,color}
\usepackage{mathrsfs}
\usepackage{float}
\usepackage{indentfirst}
\usepackage{babel}

% this is to be consistent with the newer PRB style
\usepackage[sort&compress]{natbib}
\setcitestyle{numbers,square}

\usepackage{color}

\newcommand{\bela}[1]{[\emph{\color{blue}{Bela: #1}}]}
\newcommand{\brayden}[1]{[\emph{\color{red}{Brayden: #1}}]}
\newcommand{\eqnref}[1]{Eq.~(\ref{#1})}

%\makeatletter

%%%%%%%%%%%%%%%%%%%%%%%%%%%%%%
%These may need modification
%\usepackage[section]{placeins}
\graphicspath{{../images/}{../diagrams/}{.}} %what folders to look for images in, via the command \includegraphics
\input{commands/shortcuts}
\usepackage{wasysym}
\usepackage{amsfonts}
\usepackage{booktabs}
\usepackage{subfigure}
%%%%%%%%%%%%%%%%%%%%%%%%%%%%%%%

\begin{document}

%\title{Lattice-symmetry protected edge phenomena in a featureless boson insulator}
\title{A featureless insulator with entanglement protected by lattice symmetries}
%\title{Not so featureless after all: \\ symmetry protected edge phenomena in an interacting boson state}

\author{Brayden Ware}
\affiliation{Department of Physics, University of California, Santa Barbara, CA 93106-6105, USA}

\author{Itamar Kimchi}
\affiliation{Department of Physics, University of California, Berkeley, CA 94720, USA}

\author{S. A. Parameswaran}
\affiliation{Department of Physics and Astronomy, University of California, Irvine, CA 92697, USA}

\author{Bela Bauer}
\affiliation{Station Q, Microsoft Research, Santa Barbara, CA 93106-6105, USA}

\begin{abstract}
While the Lieb-Schultz-Matthis theorem allows the existence of a fully
symmetric bosonic paramagnet at half filling per site on the honeycomb lattice, an
explicit construction of such a wavefunction is challenging and cannot
be achieved in a non-interacting system. Recently, Kimchi {\it et al.} constructed
such a wavefunction and demonstrated that it does not break any
symmetries and is not topologically ordered.
Here, we use recently developed tensor network tools to argue that this wavefunction, while
appearing completely featureless in the bulk, is not
adiabatically connected to a non-entangled state. This is achieved by exposing
non-trivial structure in the entanglement spectrum, including a gapless edge
described by a conformal field theory and degeneracies protected by the non-trivial
action of combined charge-conservation and spatial symmetries on the edge.
The phase thus constitutes an interacting, bosonic symmetry-protected phase
protected by lattice symmetries.

ALTERNATE: 

Strongly correlated analogues of topological insulators have been explored in systems with purely on-site symmetries, such as time-reversal or charge conservation. 
Here we study a quantum state of interacting bosons and show that it exhibits  entanglement properties which are protected by its spatial lattice symmetries. 
This state, a fully-symmetric insulator of bosons on the honeycomb lattice at half filling (per site), bypasses the Hastings-Oshikawa theorems through its integer unit cell filling, remaining featureless in the bulk. 
For non-interacting particles at this filling, the tight-binding description of the honeycomb lattice combines with its crystalline symmetries to trigger metallicity, as in graphene's Dirac nodes.
Here, we use recently developed tensor network tools to compute the boundary entanglement spectra and argue that this state, while featureless and insulating in the bulk, cannot be adiabatically connected to a non-entangled insulator. 
We find a gapless entanglement edge described by a conformal field theory and degeneracies protected by the non-trivial action of combined charge-conservation and spatial symmetries on the edge.
Modifying the tight-binding representation by adding sites while preserving lattice symmetries is incompatible with the edge mode entanglement cuts, suggesting that the lattice representation can serve as an additional symmetry ingredient in protecting an interacting topological phase. 

%While the Lieb-Schultz-Mattis theorem forbids the existence of fully
%symmetric quantum paramagnetic phases on lattices with fractional
%filling of particles per unit cell, such a phase is in principle
%allowed with certain fractional numbers of particles per site on
%non-Bravais lattices, including half-filling on the honeycomb lattice.
%It has been shown that a non-interacting Hamiltonian of spinless
%fermions or bosons cannot have such a symmetric insulating ground
%state, and an explicit construction using interactions is challenging.
%Recently, Kimchi et al. constructed a wavefunction for bosons at
%half-filling that does not break any symmetries and is not
%topologically ordered--and in this sense is a featureless insulator in
%the bulk. Here, however, we reveal that this wavefunction exhibits
%non-trivial structure at the edge. We apply recently developed
%techniques based on a tensor network representation of the
%wavefunction to demonstrate the presence of a gapless entanglement
%spectrum and a non-trivial action of combined charge-conservation and
%spatial symmetries on the edge. We will also discuss the possibility
%of finding a parent Hamiltonian and analyzing the existence of a
%symmetry-protected topological phase around this state.
\end{abstract}
\maketitle

%!TEX root = ../fbi.tex

\section{Introduction}

The discovery in 2007 of time-reversal invariant band insulators that are not adiabatically connected to the atomic limit~\cite{...}
has spurred the exploration of a broad array of phases where symmetries protect subtle, non-local features that distinguish
them from trivial, unentangled insulators. These phases, colectively known as symmetry-protected topological (SPT) phases~\cite{...}, have
by now been observed in several (so far, weakly-interacting) experimental realizations both in two and three dimensions~\cite{somereview} and an extensive
mathematical framework has been developed for their characterization and classification~\cite{wen...}.

While the bulk of the system in such a phase is often only subtly distinguished from a trivial insulator, the non-local or topological properties can be observed by focusing on the boundary of a finite system.  The boundary often exhibits gapless features such as the localized
Majorana zero modes at the ends of one-dimensional topological superconductors~\cite{kitaev2001},
the helical edge states of the quantum spin Hall effect~\cite{...} and the protected Dirac cones on the surface of three-dimensional
$\mathbb{Z}_2$ topological insulators~\cite{...}.
It has been shown that these features also appear in the entanglement spectrum~\cite{li2008}, where they generally take the form
of either protected degeneracies or gapless spectra that mirror the gapless modes on a boundary.
These signify entanglement that cannot be removed while preserving the symmetry, and can thus be used to establish
that SPT phases are fundamentally distinct from trivial, non-entangled insulators.

Symmetries which relate the physical locations of degrees of freedom, such as the spatial symmetries of rotation or reflection, are often observed to be fundamentaly distinct from on-site symmetries (such as charge conservation or time-reversal symmetry) in this context.  In particular, the interplay of on-site and spatial symmetries can protect new kinds of topological phases.  \cite{...}
% cite: liang fu Topological crystalline insulators, liang fu and exptalists TCI discovery,  turner etc paper on haldane chain prtected also by inversion
The role of the boundary is also modified when spatial symmetries are considered; when the symmetry may be broken by a physical boundary of a system, as for instance occurs for the case of inversion symmetry on any boundary, the non-trivial features can still be extracted from the entanglement spectrum.  The combination of entanglement and physical gapless modes thus permits topological phases to be theoretically distinguished from trivial insulators in a transparent way, even when a necessary ingredient for the topological protection is a non-local symmetry operation. 
However, while non-interacting crystalline topological insulators, whose non-trivial topology requires some non-local symmetry operations, have been well explored~\cite{...}, far less is known about their interacting counterparts in two or three dimensions. 

In this work, we compute the entanglement propeties of an insulating state of interacting bosons on the honeycomb lattice, and show that it constitutes a topological phase protected by lattice symmetries.  In particular, we show that the non-trivial entanglement is intimately related to the representation of the lattice symmetries as the honeycomb (or graphene) tight-binding model. We numerically represent the quantum state as a network of tensors, arranged to form the honeycomb lattice tight-binding model. We find that the action of charge conservation and spatial symmetries on certain entanglement cuts combines to enforce entanglement degeneracies, which remain protected as long as the tight-binding lattice is preserved.

The quantum wavefunction in the present study, first proposed in Ref.~\onlinecite{kimchi2013},
is a type of Mott insulating state which requires a non-Bravais lattice, i.e. a lattice with a non-trivial unit cell. 
The necessity for a unit cell arises due to the Lieb-Schultz-Matthis (LSM) theorem~\cite{...}.
The LSM theorem forbids the existence of a featureless state
-- a state that neither spontaneously breaks a symmetry, nor displays topological
order, nor has power-law correlations and is thus ''gapless'' -- in systems
with a fractional filling per unit cell. In contrast, the classical cartoon of a Mott insulator requires a filling which is not only integer per unit cell but also integer per site. A featureless insulating state is in principle allowed by the LSM theorem on the honeycomb lattice
at half filling per site, and thus unit filling per unit cell. Such a state is necesarily distinct from the usual integer-per-site Mott insulator. 
Moreover, despite being a state of particles at half filling on a lattice with an even number of sites per unit cell, this state is in no way related to a non-interacting band insulator.  Indeed, here no band insulator can exist: symmetry guarantees that a free-fermion spectrum is gapless at certain high-symmetry points, and there are
thus no localized Wannier basis.
This implies that a featureless state on the honeycomb lattice cannot be constructed by filling a permanent of localized Wannier
orbitals~\cite{parameswaran2013}, and any construction of a quantum state must thus involve interactions in a non-trivial way. 
Ref.~\onlinecite{kimchi2013} pursued
an approach of constructing a permanent wavefunction by filling local and symmetric
orbitals that are not orthogonal and
it was argued that that the resulting wavefunction is indeed featureless.
In particular, using numerical simulations it was found that the state exhibits
isotropic and exponentially decaying correlations, and arguments were presented that it is not
topologically ordered.

While we confirm the featureless bulk of the state, we show that nevertheless the entanglement of the state is non-trivial cannot be removed while preserving
all symmetries, i.e. it constitutes a symmetry-protected phase. The relevant symmetry is a combination of charge
conservation and lattice symmetries, which together protect universal features in the entanglement spectrum. In
particular, we show that the low-lying entanglement spectrum is to great accuracy described by that of a $c=1$
conformal field theory, and that there is an exact double 
degeneracy throughout the entanglement spectrum for certain geometries, which is protected
by the symmetries of the state and thus serves as a topological invariant identifying the SPT order.
Since lattice symmetry is involved crucially, this provides one of the first examples for an SPT of interacting bosons
protected by lattice rather than on-site symmetries.

All of these properties of the phase become accessible through a description of the state
as a projected entangled-pair state (PEPS)~\cite{verstraete2004}. These states form a specific class of tensor
network states that corresponds to a generalization of the well-known matrix-product state
(MPS)~\cite{white1992,ostlund1996,schollwoeck2010}
framework to higher dimensions. PEPS have been shown to be a powerful description of many
classes of gapped systems, including topologically ordered and SPT phases. Here, we have an
exact description of the state as a PEPS, allowing us to extract properties such as the entanglement
spectrum and the topological invariants exactly on certain geometries; we emphasize that these
properties of the state are not accessible to other numerical methods.

The topological invariants extracted here form examples of a broad class
of invariants that provably must be constant throughout the phase. These differ
from the order parameters that measure local symmetry breaking in that they
are not related to the expectation values of local operators. Early examples
of topological invariants for SPT phases are the string order parameter for the one-dimensional
AKLT phase~\cite{...}, and the spin Chern number for the quantum spin Hall effect~\cite{...}.
The invariants we consider here measure how the action of the symmetry is implemented on the 
physical edge states of open systems or on the Schmidt states of an 
entanglement decomposition~\cite{pollmann,...}. These invariants feature heavily in the 
classification of SPT phases with on-site symmetry~\cite{...}, and similar invariants 
that apply to free-fermion states have been used for topological crystalline 
insulators~\cite{bernevig...}. By contrast, topological invariants for interacting states 
protected by lattice symmetries in more than one dimension are not well 
understood. We will discuss the action of the symmetry on the edge of the  
state and progress towards the goal of finding a topological invariant to 
identify the corresponding phase.

%!TEX root = ../fbi.tex

\section{Construction of the featureless boson insulator}
\label{sec:fbi}

In the honeycomb lattice, each unit cell is associated with exactly one hexagon plaquette, which respects the lattice point group symmetries. As shown in Ref.~\onlinecite{kimchi2013}, this provides an explicit
construction of a bosonic insulator
on the honeycomb lattice that is completely featureless in the bulk,
henceforth referred to as \emph{honeycomb featureless boson insulator} (HFBI).
The state is succinctly described by the following wavefunction expression:
\begin{equation} \label{eq:def}
\ket{\psi} = \prod\limits_{\varhexagon} \left( \sum\limits_{i \in \varhexagon} b^{\dagger}_{i} \right) \ket{0}.
\end{equation}
Here, $\varhexagon$ denotes the elementary hexagons of the honeycomb
lattice. Despite the compact notation, this many-body bosonic state involves many overlaps of multiple particles and requires concrete computation for its properties to be unveiled.

We focus on two closely related variants of this state: a
version of soft-core bosons where $b_i^\dagger$ creates a boson on
site $i$ and obeys the usual bosonic commutation relations, and a
hard-core version of the same state where $b_i^\dagger$ also creates a
boson but $(b_i^\dagger)^2=0$. In either case, the operator $\sum_{i
\in \varhexagon} b^{\dagger}_{i}$ creates exactly one boson per
hexagon; as there is one hexagon per unit cell of two sites of the
lattice, the state has one boson per unit cell, or half a boson per
site, thus allowing the existence of a featureless state.
In the case of soft-core bosons, the maximum number of bosons
per site is three.

\bela{Summarize what was calculated in Ref.~\onlinecite{kimchi2013}.}

\subsection{PEPS representation}

In order to make the state~\eqnref{eq:def} more amenable to numerical
simulations, and in particular in order to be able to study its edge
properties, we now derive a representation as a projected entangled
pair state (PEPS). Importantly, this PEPS description will respect
all of the relevant symmetries of \eqnref{eq:def}.

\begin{figure}
	\centering
	\includegraphics[width=0.8\columnwidth]{Hex_PEPS.pdf}
	\caption{Honeycomb lattice PEPS and zig-zag entanglement cut.
	%A PEPS on the honeycomb lattice in the "zig-zag"' cylinder geometry used
	%througout in this paper. 
	%The cylinder is created by periodically identifying sites $W$ unit cells away 
	%- in the picture, the 
	In this PEPS of rank-4 tensors, the top and bottom edges are identified, forming a cylinder with circumference $W=3$ unit cells. 	
	%The tensors A and B are rank 4, with one physical leg and 3 virtual legs at each site. 
	A one-dimensional MPS representation is constructed by blocking the tensors on each cylinder slice (green dotted lines).
	The entanglement cut shown (bold dotted line) passes through the hexagon mid-point, preventing the tight-binding lattice from gaining additional sites as long as crystalline symmetries are preserved. 
	%creates a MPS representation for the state.
	}
	\label{fig:PEPS}
\end{figure}

To obtain a PEPS construction, we first choose a local basis $\ket{n}$
of boson occupation numbers, i.e. $b^\dagger b \ket{n} = n \ket{n}$.
The PEPS will thus describe the coefficients of $\ket{\psi}$ in this
basis, $\langle n_1 \ldots n_L | \psi \rangle$. The PEPS
representation is most easily obtained in a two-step construction,
where we first construct the state shown in Fig.~\ref{fig:FBI_PEPS}.
Here, the tensor labeled $W=W^{n_1 \ldots n_6}$, which is placed in
the center of each hexagon, is a rank-6 tensor given by
\begin{equation} \label{eq:W}
W^{n_1 \ldots n_6}  = \left\{ \begin{array}{lr}
													1  : & \sum\limits_i n_i = 1 \\
													0  : & \text{else}
													\end{array} \right.,
\end{equation}
where each $n_i \in \{0, 1\}$.

\begin{figure}
	\centering
	\includegraphics[width=0.8\columnwidth]{FBI_PEPS.pdf}
	\caption{
	Intermediate tensor network for HFBI state. Here, the tensors labeled
	$D$ are located on the sites of the honeycomb lattice, while the
	tensors labeled $W$ are located on the centers of each hexagon.
	Dotted lines thus represent the physical lattice, while the solid
	lines indicate auxiliary bonds over which the tensor network is
	contracted.}
	\label{fig:FBI_PEPS}
\end{figure}

This tensor describes the coefficients of a so-called $W$-state in the
occupation number basis, i.e. $W^{n_1 \ldots n_6}= \langle n_1
\ldots n_6 | \sum_{i=1}^6 b_i^\dagger |0\rangle$. We note that this
tensor is symmetric under permutations of its indices.

On the sites of the physical lattice, we have placed a rank-4 tensor
denoted as $D$, shown in panel (a) of Fig.~\ref{fig:FBI_PEPS_2}, which
connects the $W$ tensors from three adjacent hexagons, and as fourth
index has a physical index $p$. For a state of soft-core bosons, where
$p=0,1,2,3$, this tensor is given by
\begin{equation} \label{eqn:Dsc}
D^\mathrm{sc}_{p, i_0 i_1 i_2}  = \left\{ \begin{array}{ll}
													\sqrt{p!}  &: p =i_0+i_1+i_2  \\
													0  &:  \text{else}
													\end{array}
											\right. .
\end{equation}
We can also encode a state of hard-core bosons by replacing $D$ by
\begin{equation} \label{eqn:Dhc}
D^\mathrm{hc}_{p, i_0 i_1 i_2}  = \left\{ \begin{array}{ll}
													1  &: p = i_0+i_1+i_2 \le 1  \\
													0  &:  \text{else}
													\end{array}
											\right.
\end{equation}
Other values for the $D$ and $W$ tensors that respect the charge and lattice symmetries
can also give rise to featureless insulators. 
Some of these variants are described in Appendix~\ref{Appendix:Variants}.

This tensor network wavefunction manifestly respects all the
translational and point group symmetries of the honeycomb lattice,
since the tensors $W$ and $D$ are invariant under rotations of their
virtual indices in the plane. One can also check that the
wavefunction is $U(1)$ invariant with charge $1$ per plaquette.

\begin{figure}
	\centering
	\subfigure[D tensor]{%
		\includegraphics[width=0.18\columnwidth]{D_op.pdf}
		\label{fig:D}
	}
	\quad
	\subfigure[W-tensor and factored form]{%
		\includegraphics[width=0.5\columnwidth]{w_string.pdf}
		\label{fig:W}
	}
	\subfigure[PEPS tensor network for F.B.I. state]{%
		\includegraphics[width=0.8\columnwidth]{FBI_PEPS_2.pdf}
		\label{fig:FBI_PEPS_2}
	}
\label{fig:PEPSforFBI}
\caption{Construction of PEPS for HFBI state. 
The site tensors (shown in the shaded regions of panel (c)) are constructed
using the factors of the plaquette tensor $W$ (panel (b))
combined with the original vertex tensor $D$ (panel (a)).
The red line in panel (c) shows where the entanglement cut considered in this paper
crosses the network.
}
\end{figure}

In order to convert the tensor network of Fig.~\ref{fig:FBI_PEPS} into a PEPS representation,
we first factor the $W$-tensor into a matrix-product state
of six tensors as shown in panel (b) of Fig.~\ref{fig:PEPSforFBI}. We
choose a form of the MPS that breaks the rotational symmetry
of the W-state (which appears as translational symmetry of the MPS).
This allows us to obtain an MPS description with a small bond dimension
of $M=2$; a fully symmetric choice would require bond dimension 6.
Since these states are physically equivalent, we expect all
physical quantities to be unaffected by this choice.
One possible decomposition is given by
\begin{equation}
W^{i_1 i_2 i_3 i_4 i_5 i_6} = \sum\limits_{\alpha_1 \ldots \alpha_5} V^{i_1}_{\alpha_1} W^{i_2}_{\alpha_1 \alpha_2}
\ldots
W^{i_5}_{\alpha_4 \alpha_5} X^{i_6}_{\alpha_5}
\end{equation}
where $V^{i_1}_{\alpha_1} = \delta_{i_1, \alpha_1}$, $X^{i_6}_{\alpha_5} = \delta_{i_6,\alpha_5+1}+\delta_{i_6,\alpha_5-1}$, and
\begin{equation*}
W_{i_0 i_1}^{j}  = \left\{ \begin{array}{ll}
													1  &:  i_0+j=i_1 \\
													0  &:  \text{else}
													\end{array}
											\right.,
\end{equation*}
where each index takes values in $\{0, 1\}$. Applying this to each
$W$-tensor yields the state as shown in panel (c) of
Fig.~\ref{fig:PEPSforFBI}. By contracting the four tensors in each
shaded region together, we obtain a PEPS in the regular form as shown
in Fig.~\ref{fig:PEPS}. The resulting PEPS has a bond dimension of
$M=2$ on the horizontal bonds, and a bond dimension of $M=4$ on all
other bonds. While it superficially breaks
the rotational symmetry of the lattice, it is an exact representation
of the FBI state and does not break any symmetries after contracting
the indices.

This decomposition respects the physical U(1) charge conservation
symmetry in that all tensors are separately
U(1)-invariant~\cite{bauer2011}. To make this manifest, we have
indicated in Fig.~\ref{fig:PEPSforFBI} arrows on each bond that show
the flow of charge.

\subsection{Representation on infinite cylinders}
For the calculations presented in this manuscript, we consider the
state $\ket{\psi}$ on a cylinder of infinite length, but finite
circumference $W$. In Fig.~\ref{fig:PEPS}, we have indicated the
choice of boundary conditions for the cylinder used in this paper. 
For many practical purposes, the PEPS on an infinite cylinder can be represented as an infinite, \
translationally invariant matrix-product state of bond dimension $\chi=2^W$
and physical dimension $p=4^{2W}$ ($p=2^{2W}$) for the soft-core (hard-core) case. 
The MPS is created by blocking all tensors in each slice of the cylinder, as shown in 
Figure~\ref{fig:PEPS}.

With each cylinder slice blocked together and considered as an MPS,
the procedures we use for computing both correlation functions and entanglement properties
are in principle identical to those used previously in MPS \cite{cirac2011,pollmann2010}.
Due to the exponential increase in the MPS bond dimension, this numerically exact approach
scales exponentially in the circumference of the cylinder.
It is however computationally advantageous to exploit the additional 
structure present in the PEPS transfer operator; 
by doing so, we can compute correlation functions and
the entanglement spectrum for the cut shown in Figure \ref{fig:FBI_PEPS_2}
for the HFBI state on cylinders of circumference up to $W=10$.
These computations are presented in the following 
sections~\ref{sec:Correlations} and \ref{sec:ES}.

\input{sections/Correlations}
\input{sections/Entanglement_spectra}
\input{sections/SymmetryProtection}
\input{sections/Perturbations}
\input{sections/Conclusions}
\input{sections/Appendix}

\bibliography{fbi}

\end{document}
