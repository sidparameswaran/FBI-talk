\documentclass[twocolumn,english,prb,showpacs,superscriptaddress]{revtex4-1}
\usepackage[colorlinks=true,urlcolor=blue,citecolor=blue,linkcolor=blue]{hyperref}
\usepackage[T1]{fontenc}
\usepackage[latin9]{inputenc}
\usepackage{amssymb}
\usepackage{graphicx}
\usepackage{amsmath,color}
\usepackage{mathrsfs}
\usepackage{float}
\usepackage{indentfirst}
\usepackage{babel}

% this is to be consistent with the newer PRB style
\usepackage[sort&compress]{natbib}
\setcitestyle{numbers,square}

\usepackage{color}

\newcommand{\bela}[1]{[\emph{\color{blue}{Bela: #1}}]}
\newcommand{\brayden}[1]{[\emph{\color{red}{Brayden: #1}}]}
\newcommand{\eqnref}[1]{Eq.~(\ref{#1})}

%\makeatletter

%%%%%%%%%%%%%%%%%%%%%%%%%%%%%%
%These may need modification
%\usepackage[section]{placeins}
\graphicspath{{../images/}{../diagrams/}{.}} %what folders to look for images in, via the command \includegraphics
\newcommand{\beq}{\begin{equation}}
\newcommand{\eeq}{\end{equation}}
\newcommand{\beqa}{\begin{eqnarray}}
\newcommand{\eeqa}{\end{eqnarray}}
\newcommand{\bi}{\begin{itemize}}
\newcommand{\ei}{\end{itemize}}
\newcommand{\ket} [1] {\vert #1 \rangle}
\newcommand{\bra} [1] {\langle #1 \vert}
\newcommand{\braket}[2]{\langle #1 | #2 \rangle}
\newcommand{\ev}[1]{\langle #1 \rangle}
\newcommand{\vbra}[1]{\left ( #1 \right |}
\newcommand{\vket}[1]{\left |#1 \right )}
\newcommand{\vbraket}[2]{\left ( #1 \middle |#2 \right )} 
\newcommand{\braopket}[3]{\left \langle #1 \middle |#2 \middle | #3 \right \rangle} 
\newcommand{\vbraopket}[3]{\left ( #1 \middle |#2 \middle | #3 \right )} 

%\newcommand<>{\highlighton}[1]{%
%  \alt#2{\structure{#1}}{{#1}}
%}

\newcommand{\icon}[1]{\pgfimage[height=1em]{#1}}

\usepackage{wasysym}
\usepackage{amsfonts}
\usepackage{booktabs}
\usepackage{subfigure}
%%%%%%%%%%%%%%%%%%%%%%%%%%%%%%%

\begin{document}

%\title{Lattice-symmetry protected edge phenomena in a featureless boson insulator}
\title{A featureless insulator with entanglement protected by lattice symmetries}
%\title{Not so featureless after all: \\ symmetry protected edge phenomena in an interacting boson state}

\author{Brayden Ware}
\affiliation{Department of Physics, University of California, Santa Barbara, CA, 93106-6105, USA}

\author{Itamar Kimchi}
\affiliation{Department of Physics, University of California, Berkeley, CA 94720, USA}

\author{S. A. Parameswaran}
\affiliation{Department of Physics and Astronomy, University of California, Irvine, CA 92697, USA}

\author{Bela Bauer}
\affiliation{Station Q, Microsoft Research, Santa Barbara, CA 93106-6105, USA}

\begin{abstract}
While the Lieb-Schultz-Matthis theorem allows the existence of a fully
symmetric bosonic paramagnet at half filling per site on the honeycomb lattice, an
explicit construction of such a wavefunction is challenging and cannot
be achieved in a non-interacting system. Recently, Kimchi {\it et al.} constructed
such a wavefunction and demonstrated that it does not break any
symmetries and is not topologically ordered.
Here, we use recently developed tensor network tools to argue that this wavefunction, while
appearing completely featureless in the bulk, is not
adiabatically connected to a non-entangled state. This is achieved by exposing
non-trivial structure in the entanglement spectrum, including a gapless edge
described by a conformal field theory and degeneracies protected by the non-trivial
action of combined charge-conservation and spatial symmetries on the edge.
The phase thus constitutes an interacting, bosonic symmetry-protected phase
protected by lattice symmetries.

%While the Lieb-Schultz-Mattis theorem forbids the existence of fully
%symmetric quantum paramagnetic phases on lattices with fractional
%filling of particles per unit cell, such a phase is in principle
%allowed with certain fractional numbers of particles per site on
%non-Bravais lattices, including half-filling on the honeycomb lattice.
%It has been shown that a non-interacting Hamiltonian of spinless
%fermions or bosons cannot have such a symmetric insulating ground
%state, and an explicit construction using interactions is challenging.
%Recently, Kimchi et al. constructed a wavefunction for bosons at
%half-filling that does not break any symmetries and is not
%topologically ordered--and in this sense is a featureless insulator in
%the bulk. Here, however, we reveal that this wavefunction exhibits
%non-trivial structure at the edge. We apply recently developed
%techniques based on a tensor network representation of the
%wavefunction to demonstrate the presence of a gapless entanglement
%spectrum and a non-trivial action of combined charge-conservation and
%spatial symmetries on the edge. We will also discuss the possibility
%of finding a parent Hamiltonian and analyzing the existence of a
%symmetry-protected topological phase around this state.
\end{abstract}
\maketitle

%!TEX root = ../fbi.tex

\section{Introduction}

\bela{This is pretty clich\'{e} right now. It's also missing all references.}

Recent years have seen the proliferation of phases that defy a description
within the well-known framework of local order parameters. These phases are
not characterized by a local order parameter, but rather by more subtle,
non-local features. An important class of such phases are topological phases,
with key characteristics such as exotic, anyonic excitations. Another
important class are symmetry-protected topological phases, which have
garnered a great deal of attention since their original discovery.

Famously, the Lieb-Schults-Mattis (LSM) theorem and its extensions gives a set of
conditions under which such an exotic phase not only can, but indeed \emph{must}
exist. In particular, they forbid the existence of a featureless state
-- a state that neither spontaneously breaks a symmetry, nor displays topological
order, nor has power-law correlations and is thus ''gapless'' -- in systems
with a fractional filling per unit cell. Extensions to LSM in \onlinecite{parameswaran2013-2} further
forbid some site fractional site fillings for non-symmorphic lattices.

In Ref.~\onlinecite{kimchi2013}, an interesting special case was considered,
namely that of half-filled bosons on the honeycomb lattice. In this case, where
there is one boson per unit cell of two sites, the LSM allows the existence
of a featureless state. At the same time, however, symmetry guarantees that
a free-fermion spectrum is gapless at certain high-symmetry points and there is
thus no localized Wannier basis. This implies that
a featureless state cannot be constructed by filling a permanent
of localized Wannier orbitals, and any construction must thus involve interactions
in a non-trivial way. The explicit construction of Ref.~\onlinecite{kimchi2013},
henceforth dubbed honeycomb featureless boson insulator (HFBI),
will be discussed in detail in Section~\ref{sec:fbi}.
\bela{Possibly move the first few paragraphs of Sec II, or at least a short summary, here?}

The main focus of this paper will be on the edge properties of the HFBI.
In particular, we will find that while the state is featureless in the bulk, it
features a gapless edge state which is protected by a combination of lattice inversion
and spin symmetry. We will argue for this by calculating the entanglement spectrum,
which we achieve using recently developed tensor network methods.

The simplest example of a tensor network for a 1D system is a matrix
product state; translationally-invariant matrix product states (MPS)
are represented by a single rank 3 tensor $A_p^{ij}$ which specifies
the wavefunction coefficients in a given basis local basis as a
product of matrices $$\ket{\psi} = \sum\limits_{\{p_i\}} A_{p_1}
A_{p_2} ... A_{p_N} \ket{p_0 p_1 ... p_N},$$ with one matrix for each
site in the one-dimensional system.

PEPS are the generalization of MPS to two and higher dimensions, where
each site in the system is represented by a rank $z+1$ tensor, where
$z$ is the coordination number of the lattice, and wavefunction
coefficients are similarly given by contracting over all virtual
indices as shown in Figure \ref{fig:PEPS}.\cite{verstraete2004} 
%!TEX root = ../fbi.tex

\section{Construction of the featuresless boson insulator}

In Ref.~\onlinecite{kimchi2013}, Kimchi et al gave an explicit construction of a bosonic insulator
on the honeycomb lattice that is completely featureless in the bulk. The state is succinctly described by the following
wavefunction:
\begin{equation} \label{eq:def}
\ket{\psi} = \prod\limits_{\varhexagon} \sum\limits_{i \in \varhexagon} b^{\dagger}_{i} \ket{0}.
\end{equation}
Here, $\varhexagon$ denotes the elementary hexagons of the honeycomb lattice. We consider two closely related
variants of the state, a version of soft-core bosons where $b_i^\dagger$ creates a boson on site $i$ and obeys the
usual bosonic commutation relations, and a hard-core version of the same state where $b_i^\dagger$ also creates a
boson but $(b_i^\dagger)^2=0$.
In either case, the operator $\sum_{i \in \varhexagon} b^{\dagger}_{i}$ creates exactly one boson per hexagon;
as there is one hexagon per unit cell of two sites of the lattice, the state has one boson per unit cell, or half a boson per site.
In the case of soft-core bosons, the maximum number of bosons per site is 3.

\begin{figure}[H]
	\centering
	\includegraphics[width=0.6\columnwidth]{fbi3.pdf}
	\caption{Schematic representation of honeycomb FBI}
\end{figure}

\subsection{PEPS Construction of honeycomb F.B.I.}

In order to make the state~\eqnref{eq:def} more amenable to numerical simulations, and in particular in order to be able to study its
edge properties, we now derive a representation as a projected entangled pair states (PEPS). \bela{I would probably put the more
generic introduction to PEPS as generalization of MPS etc. in the Introduction rather than have it here.}
Importantly, this PEPS description will respect all of the relevant symmetries of \eqnref{eq:def}.

\begin{figure}[H]
	\centering
	\includegraphics[width=\columnwidth]{Hex_PEPS.pdf}
	\caption{\bela{In this picture, we need to indicate the physical legs (maybe in a different color).} Pictoral representation of a PEPS on a honeycomb lattice. The tensors A and B are rank 4, with the physical leg at each site not shown for visual clarity. By gluing together the top and bottom edge of this picture, we get the "zig-zag" cylinder configuration used for most of the calculations in this paper. This cylinder is 3 unit cells wide, so we will call it the L=3 cylinder. In section \ref{sec:ES}, we will study the entanglement cut specified by the red line for various cylinder widths.}
	\label{fig:PEPS}
\end{figure}

\begin{figure}
	\centering
	\includegraphics[width=0.8\columnwidth]{FBI_PEPS.pdf}
	\caption{\bela{Make labels math font. I think we might have to include physical indices in this figure, maybe very short/thin.}
	Intermediate tensor network for FBI state. Here, the tensors labeled $D$ are located on the
	sites of the honeycomb lattice, while the tensors labeled $W$ are located on the centers of each hexagon.
	Dotted lines thus represent the physical lattice, while the solid lines indicate auxiliary bonds over which the tensor network
	is contracted. In this picture, we have suppressed the physical index for visual clarity.}
	\label{fig:FBI_PEPS}
\end{figure}

To obtain a PEPS construction, we first choose a local basis $\ket{n}$ of boson occupation numbers, i.e. $b^\dagger b \ket{n} = n \ket{n}$. The
PEPS will thus describe the coefficients of $\ket{\psi}$ in this basis, $\langle n_1 \ldots n_L | \psi \rangle$.
The PEPS representation is most easily obtained in a two-step construction, where we first construct the state shown in Fig.~\ref{fig:FBI_PEPS}.
Here, the tensor labeled $W=W^{n_1 \ldots n_6}$, which is placed in the center of each hexagon, is a rank-6 tensor given by
\begin{equation}
W^{\{n_x\}}  = \left\{ \begin{array}{lr}
													1  : & \sum\limits_x n_x = 1 \\
													0  : & \text{else}
													\end{array} \right. .
\end{equation}
This tensor describes the coefficients of a so-called $W$-state in the occupation number basis, i.e. $W^{\lbrace n_x\rbrace }= \langle n_1 \ldots n_6 | \sum_{i=1}^6 b_i^\dagger |0\rangle$. We note that this tensor is symmetric under permutations of its indices.

On the sites of the physical lattice, we have placed a rank-4 tensor denoted as $D$ which connects the $W$ tensors from three adjacent hexagons,
and as fourth index has a physical index $p$. For a state of soft-core bosons, where $p=0,1,2,3$, this tensor is given by
\begin{equation} \label{eqn:D}
D^\mathrm{sc}_{p i_0 i_1 i_2}  = \left\{ \begin{array}{ll}
													\sqrt{p!}  &: p =i_0+i_1+i_2  \\
													0  &:  \text{else}
													\end{array}
											\right. .
\end{equation}
We can also encode a state of hard-core bosons by replacing $D$ by
\begin{equation}
D^\mathrm{hc}_{p i_0 i_1 i_2}  = \left\{ \begin{array}{ll}
													1  &: i_0+i_1+i_2 > 0  \\
													0  &:  \text{else}
													\end{array}
											\right. .
\end{equation}

%We can represent (\ref{eq:def}) as a tensor network by replacing the sum over each plaquette with a tensor contraction. The matrix elements of the plaquette boson creation operator can be specfied by a trace over one virtual qubit per site $x$ adjacent to the plaquette, as in 
%\begin{equation}
%\braopket{\{\alpha_x\}}{\sum\limits_{x \in \varhexagon} b^{\dagger}_{x}}{\{\beta_x\}} =
% \sum\limits_{\{i_x\}} W^{i_1 i_2 ... i_6} \prod\limits_x B^{i_x}_{\alpha_x \beta_x} 
%\label{eq:plaquette}
%\end{equation}
%
%in which the application of the creation operator at $x$ is controlled by the state of the virtual qubit $i_x$:
%\begin{equation*}
%B^i_{\alpha \beta} = \left\{
%     \begin{array}{lr}
%       \delta_{\alpha \beta} & : i = 0\\
%       (b^{\dagger})_{\alpha \beta} & : i = 1
%     \end{array}
%   \right\}.
%\end{equation*}
%
%The tensor $W^{i_0 i_1 i_2 i_3 i_4 i_5}$ should be taken as the W-state: 
%$$ W^{\{i_x\}}  = \left\{ \begin{array}{lr}
%													1  : & \sum\limits_x i_x = 1 \\
%													0  : & \text{else}
%													\end{array}
%											\right\} = \ket{100000} + ...\ket{000001}.
%$$
%
%By applying the operator to all plaquettes, we get a tensor network not in the form of Figure \ref{fig:PEPS}, but instead one with the rank 6 tensor $W$ located on each plaquette and a rank $3+1$ tensor $D$ located on the vertices, as shown in \ref{fig:FBI_PEPS}. 
%
%It is straightforward to show that the tensor $D^p_{ijk}$ should be
%$$
%D_{p}^{i_0 i_1 i_2} = \sum\limits_{\alpha \beta} B^{i_0}_{p \alpha}B^{i_1}_{\alpha \beta} B^{i_2}_{\beta 0}  = \left\{ \begin{array}{lr}
%													\sqrt{p!}  : & p =i_0+i_1+i_2  \\
%													0  : & \text{else}
%													\end{array}
%											\right\}
%$$
%to reproduce the action of the plaquette operators adjacent to each vertex. The boson number at the vertex $p$ can be limited to be $0, 1, 2,$ or $3$ since at most one boson comes from each of the three adjacent hexagons.

This tensor network wavefunction in this form manifestly has all the translational and point group symmetries of the honeycomb lattice, since the tensors $W$ and $D$ are invariant under rotations of their virtual indices in the plane, as shown in the picture. One can also check that the wavefunction is $U(1)$ invariant; all terms in the wavefunction are configurations with boson number exactly $1$ per plaquette.

In order to form a PEPS representation with all tensors located on the vertices, we can factor the W-state tensor and regroup the factors into PEPS tensors $A$ and $B$, as shown in Figure \ref{fig:FBI_PEPS_2}. 
There is a choice of MPS for the W-state that doesn't manifestly preserve the rotational symmetry the plaquettes, but has the small bond dimension $2$. (A different choice could be made that has bond dimension 6 but does preserve the symmetry.) This factorization of the W-state is given by

$$
W^{i_0 i_1 i_2 i_3 i_4 i_5} = \sum\limits_{abcde} W^{i_0 a}_{1} W^{i_1 b}_{a} W^{i_2 c}_{b} W^{i_3 d}_{c} W^{i_4 e}_{d} W^{i_5 0}_{e},
$$
where 
$$ W^{i_0 i_1}_{j}  = \left\{ \begin{array}{lr}
													1  : & i_0+i+1 = j \\
													0  : & \text{else}
													\end{array}
											\right\},
$$
and where each index takes values in $\{0, 1\}$. We will discover that the virtual edges in this PEPS can be assigned consistent $U(1)$ charges, with the index $i_x$ representing the charge. The arrows on the tensors show the flow of charge.

The rewriting of the tensor network does not change the physical wavefunction or physically invariant quantities, such as the Schmidt spectrum for the entanglement cut. However, it is computationally convenient - the total bond size crossed by the entanglement cut in Figure \ref{fig:PEPS} is $2^L$ for the cylinder of width $L$.

\begin{figure}[H]
	\centering
	\subfigure[W-state tensor and factored form]{%
		\includegraphics[width=0.6\columnwidth]{w_string.pdf}
		\label{fig:W}
	}
	\quad
	\subfigure[D tensor]{%
		\includegraphics[width=0.3\columnwidth]{D_op.pdf}
		\label{fig:D}
	}
	\subfigure[PEPS tensor network for F.B.I. state]{%
		\includegraphics[width=0.8\columnwidth]{FBI_PEPS_2.pdf}
		\label{fig:FBI_PEPS_2A}
	}
		
%
\caption{\bela{I think the 1 in the center of the W MPS is confusing. Also, we should have shaded circles showing what the final PEPS tensor is.} Tensors used to put the F.B.I. tensor network in PEPS form. }
\label{fig:FBI_PEPS_2}
\end{figure}

%\begin{figure}[H]
%	\begin{columns}{t}
%	\begin{column}{width=0.25\textwidth]
%	\includegraphics[width=0.6\columnwidth]{w_string.pdf}
%	\end{column}
%	\begin{column}{width=0.25\textwidth]
%	\includegraphics[width=0.6\columnwidth]{D_op.pdf}
%	\end{column}
%	\end{columns}
%\end{figure}

%!TEX root = ../fbi.tex

\section{Correlation functions}

\begin{itemize}
\item Show correlation bounds for soft-core and hard-core states, correlation function fits.
\item Emphasize that $\xi < L$ for the attainable system sizes $L$, so that we can have some trust in our results.
\item Show that correlations are isotropic.
\item Show that everything agrees with MC results, as far as those are available.
\end{itemize}

\begin{figure}[H]
	\centering
		\includegraphics[width=\columnwidth]{TransferMatrixSpectrum_big.pdf}
\end{figure}


%!TEX root = ../fbi.tex

\section{Entanglement spectrum}
\label{sec:ES}

The quasi-1D cylinder geometry is also convenient for calculating the 
entanglement spectrum for entanglement cuts transverse to the 
cylinder. With each cylinder slice blocked together and considered as an MPS, the procedure for computing the entanglement spectrum is identical to those used for MPS. The process also results in a change of basis on the virtual legs that allows one to change the tensor network in Figure \ref{fig:FBI_PEPS} into a canonical form MPS,

$$
\ket{\psi} = \sum\limits_{\{p_i\}} \ldots \Lambda \Gamma_{p_0} \Lambda \Gamma_{p_1} \Lambda \Gamma_{p_2} \Lambda \ldots \ket{... p_0 p_1 p_2 ...}.
$$

Each physical leg represents all $2L$ physical indices of a cylinder slice, 
and each virtual leg represents all $L$ virtual indices that connect cylinder 
slices. The change of basis generally mixes the Hilbert spaces from these 
virtual legs, so the resulting basis won't be local around the circumference 
of the cylinder.
For many of the calculations that follow, we'll use this 1D form, and the 
methods will be entirely MPS methods. 

As is well known, the key property of the canonical form is that it provides a Schmidt decomposition $\ket{\psi} = \sum\limits_i \ket{\psi_L^i} \lambda_i \ket{\psi_R^i}$, as shown in Figure \ref{fig:mps_canonical}. The Schmidt vectors $\ket{\psi_{L/R}^i}$ make an orthonormal set of eigenvectors for the reduced density matrices $\rho_{L/R}$ of the left/right half of the system. 

\begin{figure}[htbc]
	\centering
		\includegraphics[width=0.8\columnwidth]{{mps_canonical.pdf}}
		\caption{}
		\label{fig:mps_canonical}
\end{figure}

This Schmidt decomposition only has a finite number $\chi$ of contributing 
terms - one for each non-zero eigenvalue $\rho_i = \lambda_i^2$ of the reduced 
density matrices on either side of the cut -
which is bounded by the total bond dimension crossed by the 
entanglement cut.  For the HFBI and the cut shown in Figure 
\ref{fig:FBI_PEPS}, the number of terms is $2^L$. 

The entanglement Hamiltonian $H_e$ is defined via  $e^{-H_e} = \rho$,
where $\rho$ is the reduced density matrix for the half-infinite cylinder
on one side of the entanglement cut. Its eigenvectors are thus also the Schmidt
states, and its eigenvalues are $\epsilon_i =-\log \rho_i$. 
Due to the finite correlation length of the state, the
differences between the Schmidt states are exponentially localized to the 
boundary. For this reason, we refer to these states as the `entanglement 
edge'. 
 
As shown in \cite{perezgarcia2008}, a translationally invariant MPS can be
assigned a projective representation of the on-site symmetry group, acting on 
the virtual legs, that maps Schmidt states to degenerate Schmidt states. 
For the MPS formed by the HFBI, we find that the U(1) boson number symmetry 
and the$\mathbb{Z}_L$ translational symmetry around the cylinder circumference 
are represented linearly. Each Schmidt state can thus be assigned a 
well-defined quantum number of charge and transverse momentum. We'll discuss 
this assignment of charge further in \ref{sec:symmetry}.
 
The entanglement spectra for the HFBI on cylinders with even and odd width 
circumferences are shown in \ref{fig:ESL10} and \ref{fig:ESL9}, plotted 
against the transverse momentum eigenvalue and colored by the U(1) charge 
eigenvalue of the corresponding Schmidt states. 
We find that the entanglement spectrum looks like it has a gapless
edge mode with linear dispersion near momentum zero.

\begin{figure}[htbc]
	\centering
	\includegraphics[width=\columnwidth]{{EntanglementSpectrum_L10.pdf}}
	\caption{Entanglement spectrum on a zig-zag edge L=10 cylinder}
	\label{fig:ESL10}
\end{figure}


\begin{figure}[htbc]
	\centering
	\includegraphics[width=\columnwidth]{{EntanglementSpectrum_L9.pdf}}
	\caption{Entanglement spectrum on a zig-zag edge L=9 cylinder}
	\label{fig:ESL9}
\end{figure}

In figure \ref{fig:EEScaling}, we compare the lowest points in 
the spectra for several cylinder widths. The finite size scaling confirms that 
the entanglement gap closes as $1/L$, as you would expect for a gapless mode 
with linear dispersion.

\brayden{Something about history of 2d SPTs having gapless 
entanglement edges. Possibly see brayden's advancement talk.}

Gapless entanglement spectra have been shown to be robust features of 
two-dimensional topological and symmetry protected topological (SPT) phases.

\begin{figure}[htbc]
	\centering
	\includegraphics[width=\columnwidth]{{EntanglementEnergyScaling.pdf}}
  \caption{Power law fits for the lowest five states above the ground state in Figure \ref{fig:ESL10}. The 
  $1/L$ scaling is a signature of a gapless (entanglement) Hamiltonian. Due to the small system size, 
  points at non-zero momentum still deviate significantly from $1/L$ scaling.}
  \label{fig:EEScaling}
\end{figure}

\begin{figure}[htbc]
	\centering
  \includegraphics[width=\columnwidth]{{TopologicalEntanglementEntropy.pdf}}
	\caption{The topological entanglement entropy $\gamma$ is consistent 
	with 0.}
	\label{fig:TopologicalEE}
\end{figure}

The most likely explanation for this gapless edge mode is that the state has 
topological or SPT order. By computing the topological entanglement entropy, 
as shown in \ref{fig:TopologicalEE}, we can rule out 
topological order for this state. Combined with the vanishing of topological 
entanglement entropy, these spectra suggest that the 
HFBI state are in a featureless SPT phase. 


However, the entanglement spectrum of this one wavefunction is not enough to 
determine the SPT nature of the phase. First, it isn't clear what group or 
groups of symmetries can be used to protect the gapless edge. It could be that 
the entire symmetry group - $U(1)$ boson number conservation, \brayden{time 
reversal symmetry?}, as well as the entire rotational and translational group 
of the lattice - must be preserved to protect the edge. Or, as we will argue 
is the case, a much smaller subgroup could be used to protect the edge. This 
leads to a wider class of perturbations that leave the edge intact - and since 
we will argue that translation is not needed for protection, the gapless edge 
will additonally be robust to some types of weak disorder and can be seen in 
small systems with symmetry preserving boundary conditions. Second, the 
entanglement spectrum alone fails to distinguish between other SPT phases with 
the same protecting group - for that, we need a topological invariant. Third, 
we would like to confirm the SPT nature of the phase by perturbing a parent 
Hamiltonian with terms that break various combinations of symmetries and 
seeing if they destroy the gapless edge.

%To determine the protecting group, we can proceed in 
%two ways: we could perturb a parent Hamiltonian with terms that break various 
%combinations of symmetries and see which perturbations destroy the gapless 
%edge, or we can look for a topological invariant by analyzing the action of 
%the symmetry of the entanglement edge. We will take up the later question 
%first, then return to the prospect of perturbing a parent Hamiltonian in 
%Section \ref{sec:perturbations}.

Notice that many points in the spectrum are doubly degenerate - those that are 
assigned a non-zero $U(1)$ charge - and on odd circumference cylinders, the 
entire entanglement spectrum is doubly degenerate, a property shared with 1D SPTs. This suggests that the the HFBI state as a 1D state with any  
fixed odd cylinder circumference $L$ is a 1D SPT. We will discuss in
Section \ref{sec:symmetry} how to explain the degeneracies in these spectra 
using the action of the symmetry on the edge, and how to prove that the odd 
circumference cylinder states are indeed 1D SPTs while the even circumference 
cylinder states are not. This will shed light on what the appropriate symmetry 
group to use.

Given that only odd circumference cylinders are SPTs, one might wonder whether 
odd and even $L$ cylinders approach two differt phases in the thermodynamic 
limit. In Section \ref{sec:CFT}, we will provide evidence against that 
possibility, by showing that in both cases, the low-energy, linear dispersing 
part of the entanglement spectra can be described by the same conformal field 
theory. Thus, in the thermodynamic limit, the two edge spectra approach the 
same set of points.
	
\subsection{Symmetry action on the edge}
\label{sec:symmetry}

When we treat the HFBI wavefunction on a 
cylinder of a fixed circumference $L$ as a matrix product state, we can use 
MPS methods to determine the action of an on-site symmetry on 
the virtual legs of this MPS. This is done using the MPS canonical form, as 
originally detailed in \cite{perezgarcia2008}. This will allow us to determine 
the boson charge and transverse momentum associated with each Schmidt state, 
used in the spectra plots Figures \ref{fig:ESL10} and \ref{fig:ESL9}.

This symmetry action is determined on the Schmidt basis, but we can also 
change back to the basis determined by the virtual legs of the PEPS in Figure 
\ref{fig:FBI_PEPS_2}, i.e. a basis that is local around the circumference of 
the cylinder. We'll use this basis to describe the edge action of the on-site 
symmetries of the MPS: U(1) charge, translation, and reflection $I_y$. 

Each on-site symmetry of the wavefunction $U_g = \otimes_i u^i_g$, with $U_g 
\ket{\psi} = e^{i \Theta_g} \ket{\psi}$ is assigned an operator $V_g$ that 
acts on the virtual leg of the MPS, that satisfies the equation
\begin{center}
\includegraphics[width=0.6\columnwidth]{group_sym.pdf}
\end{center}

For an MPS with a nondegenerate largest 
transfer matrix eigenvalue, this equation is guaranteed have a unique solution 
for $V_g$, and the operators $V_g$ are guaranteed to commute with the diagonal 
matrix $\Lambda$. However, these operators are not guaranteed to form a linear 
representation of the group of on-site symmetries, but in general could make 
up a projective representation, satisfying 
$$V_g V_h = e^{i \omega(g, h)} V_{gh}.$$



%By combining $N$ such blocks together and connecting them with periodic 
%boundary conditions, we recover the fact that 
%$U_g \ket{\psi} = e^{i \Theta_g} \ket{\psi}$,
%with $\Theta_g = N \theta_g$, since all of the $V_g$ and $V_g^{\dagger}$s 
%cancel pairwise.
%$\theta_g$ measures the charge per unit cell
    

%In general, the $V_g$ do not have to form a linear representation of the 
%group, but could instead be a projective representation. This can be explained 
%in the MPS language as follows: if instead the blocks are connected together 
%with different boundary conditions, as in Figure \ref{fig:mps_boundary}, we 
%discover that $U_g$ transforms $\ket{\psi}$ to a similar state with 
%transformed boundary conditions, $B \rightarrow V_g^{\dagger} B V_g$. This 
%action on $B$, one can show, is a group representation.

%For open ends - if $B$ factors into left and right parts, $B = 
%\vket{b}\vbra{b}$ - the action of the symmetry on the boundary can also be 
%factored, $\vket{b} \rightarrow V_g \vket{b}$. This action does not have to be 
%a group representation, but can fail up phases $V_g V_h = e^{i \omega(g, h)} 
%V_{gh}$.  

%\begin{figure}[htbc]
%    \centering
%    \includegraphics[width=\columnwidth]{mpsbc.png}
%    \caption{Matrix Product State with arbitrary boundary conditions}
%    \label{fig:mps_boundary}
%\end{figure}

An extension of this to treat inversion symmetry is explained in 
\onlinecite{pollmann2010}. It works identically, but instead each matrix 
product state is transposed in the basis \brayden{Add pictures for inversion 
symmetry in MPS}.



\begin{tabular*}{\columnwidth}{@{\extracolsep{\stretch{1}}}*{5}{r}@{}}
\toprule
$\mathbf{G}$ & $\mathbf{U_g}$ & $\mathbf{\theta_g}$ & $\mathbf{V_g}$ &$\mathbf{V_g V^*_g}$ \\
\midrule
 $U(1) $ & & & & \\
 $\mathcal{\pi} \mathcal{I}_x$ & & & & \\
 $\mathcal{I}_x \mathcal{I}_y$ & & & & \\
 $\mathcal{\pi} \mathcal{I}_x \mathcal{I}_y$ & & & & \\
\bottomrule
\end{tabular*}

Since 
$$  
V_{\mathcal{\pi} \mathcal{I}_y} V_{\mathcal{\pi} \mathcal{I}_y}^* = -I \text{\quad or \quad } V_{\mathcal{\pi}} V_{\mathcal{I}_y} = - V_{\mathcal{I}_y} V_{\mathcal{\pi}},
$$ 

the representation is in the nontrivial class of 

$$
H^2(\mathbb{Z}_2 \times \mathbb{Z}_2^{\mathcal{I}}; U(1)) = \mathbb{Z}_2.
$$


\newcommand{\uL}{\mathbf{L_0}}
\newcommand{\bL}{\mathbf{\bar{L}_0}}

\subsection{Identification of edge CFT}
\label{sec:CFT}

Given the $U(1)$ symmetry of the state, the simplest possible 
conformal field theory we might expect to appear at the edge is that 
of a single free bosonic field. 

The free-boson CFT is created from the Lagrangian 
$$ \mathfrak{L} = \frac{g}{2}\int dt \int\limits_0^L dx ( \frac{1}{v^2}(\partial_t \phi)^2 - (\partial_x \phi)^2)$$
and with the compatified field identification
$$ \phi \equiv \phi + 2\pi R$$
and placed on the circle of circumference $L$ with periodic boundary conditions
$$ \phi(x) \equiv \phi(x+L).$$

After canonical quantization, it is found that the set of energy 
eigenstates consists of $U(1)$ Kac-Moody primaries $\ket{e, m}$, with 
integers $e, m$ labeling the $U(1)$ charge and the winding number of 
the bosonic field respectively, and level $n, \bar{n}$ descendant 
fields for each primary - such as  $\mathbf{\bar{j}}_{-\bar{n}} 
\mathbf{j}_{-n} \ket{e, m}$ - for any nonnegative integers $n, 
\bar{n}$. The number of level $n, \bar{n}$ descendants of a given 
primary, all of which are degenerate, is $Z(n) Z(\bar{n})$, where 
$Z(n)$ is the number of partitions of the integer $n$.

The properties of the $U(1)$ Kac-Moody algebras constrain the form of 
energy and momentum eigenvalues - for the state 
$\mathbf{\bar{j}}_{-\bar{n}} \mathbf{j}_{-n} \ket{e, m}$, 

\begin{align*}
	\mathbf{P} =\frac{2\pi}{L}&(\uL-\bL) 
	&=& \frac{2\pi}{L}(em + n - \bar{n}) \\
	\mathbf{H} = \frac{2\pi}{L}&(\uL+\bL) 
	&=& \frac{2\pi}{L}(\frac{\kappa e^2}{2} + \frac{m^2}{2 \kappa} + \frac{n + \bar{n}}{2}) %\\
\end{align*}

By rescaling the energy and momentum, we find a system size 
independent pattern that can be matched to the low-energy, linearly 
dispersing part of the entanglement spectrum from Figures 
\ref{fig:ESL10} and \ref{fig:ESL9}. 

\begin{align*}
\mathbf{P} &\propto (em + n - \bar{n}) \\
\mathbf{H} &\propto e^2 + \frac{m^2}{\kappa^2} + \frac{1}{\kappa}(n + \bar{n})
\end{align*}

The label $m$ is 0 for all states in the linearly dispersing cone 
around $K=0$ - however, the primary states $\ket{e, m=\pm 1}$ can be 
found centered around momentum $K=\pi$, as seen in critical spin 
chains [reference]. The states with nonzero $e$ and zero $m$ are 
degenerate in energy and momentum with the same state with charge 
$-e$, but the $Z(n)Z(\bar{n})$ degeneracies predicted for descendant 
states are split by finite size effects.

The parameter $\kappa$ that appears - related to the coupling constant 
$g$ in the effective Lagrangian - is free. Quantum models that exhibit 
critical points that show the behavior of the free-boson CFT in fact 
have a whole line of critical points with varying values of $\kappa$, 
which can be tuned using a marginal operator in the theory.

\begin{figure}[htbc]
	\centering
	\includegraphics[width=\columnwidth]{EEIdentify.pdf}
	\caption{The identification of the primary states $\ket{\pm e, m=0}$ and the level $n, \bar{n}$ 
	descendants in the spectrum of the soft-core boson entanglement Hamiltonian. The states are labeled 
	$e_{n, \bar{n}}$. The zero and scale of the numerical spectrum are set by matching the lowest two 
	states. The energies and charges of the primaries with charges $2, 5/2, ... 4$ appear at the predicted 
	spots.  The best estimate for the Luttinger parameter from this spectrum is $\kappa \approx 1/6.4$, 
	taken from the energy of the $0_{1, 0}$ state. }
	\label{fig:EEIdentify}
\end{figure}

We can also take the lowest-lying Schmidt state, interpret it as the 
ground state of a 1d Hamiltonian, and consider its entanglement.

\begin{figure}[htbc]
	\centering
	\includegraphics[width=\columnwidth]{{EdgeGS_EntanglementEntropy.pdf}}
	\caption{Entanglement entropy within the entanglement ground state 
of the soft-core boson state on $10$ sites. For comparison, the 
Cardy-Calabrese formula $S(x) = c/3 \log \sin( \pi x/L) + const.$ is 
shown with $c=\frac{1}{2}, 1,$ and $2$, with the $const.$ fixed by 
matching the maximum of the entanglement entropy data. $c=1$ is a good 
fit.}
	\label{fig:EdgeGS_EE}
\end{figure}
%!TEX root = ../fbi.tex

\section{Symmetry protection}
\label{sec:symmetry}

\subsection{Overview}

While the gapless entanglement spectrum observed above is consistent with a symmetry-protected
topological phase, it does not by itself guarantee the presence of such a robust phase, and does not
allow us to interfere which symmetries are protecting the topological properties of the phase.
A key observation that allows us to make progress on these crucial questions is that many points
in the entanglement spectrum are degenerate. In particular, we find that for cylinders of odd circumference,
the entire spectrum is doubly degenerate.
In this section, we will discuss how
the corresponding degenerate Schmidt states are related through the action of a symmetry of the HFBI wavefunction. 
As discussed in Ref.~\onlinecite{pollmann2010} and reviewed in the Appendix~\ref{Appendix:MPS},
this symmetry action can be used to diagnose one-dimensional symmetry protected topological order,
for which the degeneracy throughout the entire entanglement spectrum is a robust feature.
We will demonstrate that the odd circumference cylinders, considered as quasi-one-dimensional states, 
are indeed SPTs protected by a combination of lattice inversion and charge parity symmetries.

While the Schmidt eigenstates are uniquely defined for non-degenerate eigenvalues of the reduced
density matrix, they are not unique when the spectrum is degenerate and any choice of orthonormal
states in the degenerate subspace represents a valid choice of Schmidt states. Applying
a unitary transformation $V^{ji}$, which respects $\sum_i V^{ji} (V^{ki})^* = \delta_{jk}$, on the
left Schmidt states must be accompanied by an appropriate transformation $(V^{ji})^*$ applied to
the right Schmidt states.

In particular, this allows the action of an on-site symmetry (or more generally, 
any symmetry which commutes separately with the reduced density matrices
for the left and right half) to mix Schmidt states corresponding to degenerate eigenvalues.
The action of such a symmetry operator $U_g$ takes the form
\beq
\label{eq:symschmidt}
\begin{split}
U_g \ket{\psi^{(i)}_{L}} &= \sum\limits_j \ket{\psi^{(j)}_{L}} V_g^{ji} \\
U_g \ket{\psi^{(i)}_{R}} &= \sum\limits_k \ket{\psi^{(k)}_{R}} \left(V_g^{ki} \right)^*,
\end{split}
\eeq
where the $V_g^{ji}$ are unitary matrices that only act on degenerate blocks of Schmidt states.
Crucially, Ref.~\onlinecite{pollmann2010} describes a numerical procedure to calculate $V_g$ for
an on-site symmetry $g$ within the MPS formalism, which we review in Appendix~\ref{Appendix:MPS}.

We can also analyze the effects of symmetry that preserve the entanglement cut but swap
the left and right of the cylinders. In general, we will consider 
any symmetry $h$ that swaps the cylinder sides and squares to the identity, which we will call an 
inverting symmetry. These satisfy a modification of~\eqnref{eq:symschmidt}:
\beq
\label{eq:isymschmidt}
\begin{split}
U_{h} \ket{\psi^{(i)}_{L}} &= \sum\limits_j \ket{\psi^{(j)}_{R}} V_{h}^{ji} \\
U_{h} \ket{\psi^{(i)}_{R}} &= \sum\limits_k \ket{\psi^{(k)}_{L}} \left( V_{h}^{ki} \right)^*.
\end{split}
\eeq
Note here that the left and right Schmidt states are exchanged in the transformation.
The coefficients $V_h$ can be used to define a symmetry action on the right Schmidt states
\beq
\label{eq:isymschmidt2}
V_{h} \ket{\psi^{(i)}_{R}} = \sum\limits_j \ket{\psi^{(j)}_{R}} V_{h}^{ji}
\eeq
where $V_{h} = U_{h} S$ and 
\beq
S \ket{\psi^{(i)}_{R}} = \ket{\psi^{(i)}_{L}}.
\eeq
Crucially, the map $S$ that comes from the Schmidt pairing is antiunitary, since
a change in phase $\ket{\psi^{(i)}_{R}} \to e^{i \varphi} \ket{\psi^{(i)}_{R}}$ must be
accompanied by the complex conjugate $\ket{\psi^{(i)}_{L}} \to e^{-i \varphi} \ket{\psi^{(i)}_{L}}$
to preserve the Schmidt decomposition. Therefore $V_h$ is antiunitary.
Together with the requirement that the symmetry squares to the identity, one finds that
(where $\mathbf{K}$ represents complex conjugation in the canonical basis)
\begin{equation}
V_h V_h^* = (V_h \mathbf{K})^2 = e^{i \phi_h} I = \pm I,
\end{equation}
that is the inverting symmetry forms an anti-unitary projective representation of $\mathbb{Z}_2$.

As reviewed in Appendix~\ref{Appendix:MPS}, the collection of $V_g$ of on-site symmetries 
sometimes fail to satisfy the group multiplication laws, i.e. one may find $V_{g_1 g_2} \neq V_{g_1} V_{g_2}$.
Instead, they may form a projective representation, where group multiplication laws are obeyed up
to phases $\omega(g_1, g_2)$, i.e. $V_{g_1} V_{g_2} = \omega(g_1, g_2) V_{g_1 g_2}$.
The equivalence classes of $\omega(g_1, g_2)$ are classified by the elements of $H^2(G, U(1))$. 
If the element is non-trivial, i.e. there exist $g_1$, $g_2$ such that $\omega(g_1,g_2) \neq 1$,
the degeneracy in the entanglement spectrum on which $V_g$ acts cannot be removed
without breaking the symmetry, since the distinct classes of $H^2(G, U(1))$ cannot be
connected continuously. \bela{This is not obvious and needs a citation.}
Similarly, for the inverting (anti-unitary) symmetries $h$, the phase $\phi_h = -1$ signifies that 
the degeneracy cannot be removed without breaking the symmetry.\cite{pollmann2010}.

\subsection{Symmetry protection of the HFBI}

The on-site symmetries of the featureless boson insulator considered here
are the $U(1)$ charge symmetry and the 
anti-unitary symmetry $\tau$, which acts by complex conjugation in the boson number basis.
Despite being at half-filling, the hard-core boson variant of the state does not have a particle-hole
symmetry. Exploring the edge action of these symmetries numerically, we find that they are all
represented linearly and thus do not protect the degeneracy of the entanglement spectrum
on cylinders of odd circumference.
In order to protect the degeneracy, we must therefore include lattice symmetries.

%In the cylinder geometry chosen, the group of lattice symmetries consists of translations 
%- generated by $T_x$ parallel and $T_y$ perpendicular to the cylinder axis-
%as well as reflections $\I_x$ through a line parallel 
%and $\I_y$ through a line perpendicular to the cylinder axis. 
%We also consider lattice inversion $\I = \I_x \I_y$, equivalent to a $\pi$ 
%rotation of the spatial plane about the center of a hexagonal plaquette.
By choosing a cylinder geometry, we explicitly break some of the lattice rotational
and reflection symmetries. In this case, we will only need the lattice inversion $\I$,
a $\pi$ rotation of the spatial plane about the center of a hexagonal plaquette. 
By analysing Equation~\ref{eq:isymschmidt2}, one sees that $V_{\I}$ will map each Schmidt 
state to a state with the same entanglement energy and momentum, but with opposite charge -
since $U_{\I}$ flips momentum, but the Schmidt pairing $S$ pairs each state with a state of
opposite charge and momentum. 
Thus $V_{\I}$ swaps the degenerate pairs of states $\ket{e, m; n, \bar{n}}$ and
$\ket{-e, -m; \bar{n}, n}$ which appear in Figure~\ref{fig:ESL910}. 

Our numerical results show that 
$$
V_{\I} V_{\I}^* = I, 
$$
so inversion alone does not protect the degeneracy. Instead, we find that the
combined symmetry $\varPi \mathcal{I}$ protects the degeneracy:
$$
V_{\varPi \I} V_{\varPi \I}^* = -I 
$$
where $\varPi = e^{i \pi \mathcal{Q}} \in U(1)$ is the charge parity symmetry.
This can be understood simply. For odd $W$, where the charges $e$ are half-integer,
states with opposite charge $e$ and $-e$ also differ in charge parity. $V_{\varPi}$
acts with a relative phase between these states. Thus, if $V_{\I}$ squares
to $I$, $V_{\varPi \I}$ squares to $-I$, and vice versa.  
It is easy to produce a variant on the HFBI state with the opposite situation where
the roles of $\I$ and $\varPi \I$ are switched - this is discussed further in 
Appendix~\ref{Appendix:Variants}.

\brayden{What do you think of this discussion?}
In the next section, we will discuss the consequences of this 1D symmetry protection 
for perturbations of a quasi-local parent Hamiltonian. Most importantly, it tells us
that the edge physics is not finely tuned but survives perturbations that respect the
protecting symmetry; specifically, perturbations of the Hamiltonian that 
remain on the honeycomb lattice,
are invariant under $\varPi \I$,
and have unique ground states on odd $W$ cylinders
will also show the degenerate entanglement spectrum in their ground states.
These conditions are enough to show that the state is a 2D SPT - that is, 
it cannot be adiabatically connected to a product state - when both $\varPi \I$ and
the translationally symmetry group of the lattice are preserved. Not breaking the
translational symmetry group is necessary for the argument to go through, since spontaneous
translational symmetry breaking leads to states that do not have unique ground states
on odd $W$ cylinders - and explicit breaking of translational symmetry, say by adding sites
off the honeycomb lattice, can blur the distinction between odd and even $W$ cylinders.
On the other hand, this symmetry protection does not allow us to guarantee the thermodynamic
degeneracy of the entanglement spectrum - such as a gapless entanglement edge - for regions
that don't wrap around the cylinder or on even cylinder sizes. If a gapped and 2D local 
parent Hamiltonian can be found, it is likely that arbitrary shaped entanglement cuts
must show entanglement features if enough symmetry is protected - but the topological invariant 
given is not enough evidence for that, and it is not clear how much symmetry will be needed.

%Using the techniques of MPS, we can find a quasi-local parent Hamiltonian for a given
%fixed cylinder width $W$ that acts on neighboring cylinder slices. By adding perturbations
%to this Hamiltonian on a cylinder with $W=3$, we can check that the degeneracy of the
%entanglement spectrum does not get split even when charge and translational
%symmetry are explicitly broken, as long as $\varPi \I$ is unbroken. In addition, this
%symmetry protection implies a non-local string order parameter that detects this phase.

There is also a second symmetry group that can protect the entanglement degeneracy.
Since $V_{\tau}$ and $V_{\I}$ acts antiunitarily, one can show that $V_{\tau \I}$ acts
unitarily on the edge. The $\mathbb{Z}_2 \times \mathbb{Z}_2$ group generated by 
$\Pi$ and $\tau \I$ to act projectively on the edge. The relavant topological invariant
is given by 

\beq
V_{\varPi} V_{\tau \I} V_{\varPi}^{-1} V_{\tau \I}^{-1}
 = - I.
\eeq


\begin{tabular*}{\columnwidth}{@{\extracolsep{\stretch{1}}}*{6}{r}@{}}
\toprule
$\mathbf{G}$ & i & Q & K & $\mathbf{\theta_g}$ & $\mathbf{\phi_g}$ \\
\midrule
 $Id            $ & + & + & + & + & + \\
 $\varPi        $ & + & + & + & - & + \\
 $\I            $ & - & - & + & + & + \\
 $\tau          $ & - & + & - & + & + \\
 $\varPi \I     $ & - & - & + & - & - \\
 $\varPi \tau   $ & - & + & - & - & + \\
 $\tau \I       $ & + & - & - & + & + \\
 $\varPi \tau \I$ & + & - & - & - & - \\
\bottomrule
\end{tabular*}



%However, the entanglement spectrum of this one wavefunction is not enough to
%determine the SPT nature of the phase. First, it isn't clear what group or
%groups of symmetries can be used to protect the gapless edge. It could be that
%the entire symmetry group - $U(1)$ boson number conservation, \brayden{time
%reversal symmetry?}, as well as the entire rotational and translational group
%of the lattice - must be preserved to protect the edge. Or, as we will argue
%is the case, a much smaller subgroup could be used to protect the edge. This
%leads to a wider class of perturbations that leave the edge intact - and since
%we will argue that translation is not needed for protection, the gapless edge should additonally be robust to some types of weak disorder and can be seen in
%small systems with symmetry preserving boundary conditions. Second, the
%entanglement spectrum alone fails to distinguish between other SPT phases with
%the same protecting group - for that, we need a topological invariant. Third,
%we would like to confirm the SPT nature of the phase by perturbing a parent
%Hamiltonian with terms that break various combinations of symmetries and
%seeing if they destroy the gapless edge.

%To determine the protecting group, we can proceed in
%two ways: we could perturb a parent Hamiltonian with terms that break various
%combinations of symmetries and see which perturbations destroy the gapless
%edge, or we can look for a topological invariant by analyzing the action of
%the symmetry of the entanglement edge. We will take up the later question
%first, then return to the prospect of perturbing a parent Hamiltonian in
%Section \ref{sec:perturbations}.


%Given that only odd circumference cylinders are SPTs, one might wonder whether
%odd and even $L$ cylinders approach two differt phases in the thermodynamic
%limit. In Section~\ref{sec:CFT}, we will provide evidence against that
%possibility, by showing that in both cases, the low-energy, linear dispersing
%part of the entanglement spectra can be described by the same conformal field
%theory. Thus, in the thermodynamic limit, the two edge spectra approach the
%same set of points.
%!TEX root = ../fbi.tex

\section{Quasi-local parent Hamiltonian and perturbations}

\label{sec:perturbations}

This could be achieved for example by adding a staggered field 
$$
H' = h \sum\limits_{i} (-1)^i \left(b_i + b_i^{\dagger}\right),
$$
where the coefficients $(-1)^i$ takes opposite values on the two sublattices, 
to a Hamiltonian for the state. More complicated staggering fields that break translational 
symmetry but change sign under inversion can be also be used - additionally, the symmetry $\I_y$ 
can be used in place of $\I$ which protects against different sets of patterns of staggering the 
field.


%!TEX root = ../fbi.tex

\section{Conclusion and Discussion}

\appendix

\section{Determining the edge action of the symmetry using MPS}
\label{Appendix:MPS}

We can use the formalism of MPS to assign an action
of the symmetry on the Schmidt states, and in particular for the charge and 
translation on-site symmetries the Schmidt states can be simultaneously assigned
charge and translation quantum numbers. This method of discovering the symmetry 
action will reproduce the action discussed in \ref{sec:ES} and quantum numbers used
in the spectra plots shown in Fig.~\ref{fig:ESL910}.
  
Let's discuss this formalism briefly. In addition,
we will discuss a generalization of this method that allows us to numerically
determine the symmetry action of inversion symmetry on the Schmidt states,
as in Section~\ref{sec:symmetry}. 
Both of these discussions follow Ref.~\onlinecite{pollmann2010}.

These discussions start by finding tensors $\Gamma, \Lambda$ representing the 
so-called canonical form of the MPS, as
originally detailed in \cite{perezgarcia2008}. This canonical form provides the Schmidt
decomposition at each site in the lattice.
\beq
\ket{\psi} = \sum\limits_{\{p_i\}} \ldots \Lambda \Gamma_{p_0} \Lambda \Gamma_{p_1} \Lambda \Gamma_{p_2} \Lambda \ldots \ket{... p_0 p_1 p_2 ...}.
\eeq
As a reminder, each physical leg represents all $2W$ physical sites on a cylinder slice,
and each virtual leg represents all virtual indices that connect cylinder slices.
The change of basis to canonical form generally mixes the Hilbert spaces from these
virtual legs, so the resulting basis won't be local around the circumference
of the cylinder.

Each on-site symmetry of the wavefunction $U_g = \otimes_i u^i_g$, with $U_g
\ket{\psi} = e^{i \Theta_g} \ket{\psi}$ is assigned an operator $V_g$ that
acts on the virtual leg of the MPS, that satisfies the equation
\begin{center}
\beq
\label{eq:onsitesym}
\eeq
\vskip-5em
\includegraphics[width=0.6\columnwidth]{group_sym.pdf}.
\end{center}
This equation can be rewritten and solved as an eigenvector problem;
for an MPS with a nondegenerate largest transfer matrix eigenvalue,
this equation is guaranteed have a unique solution where the eigenvalue $e^{i \theta_g}$
is the largest eigenvalue of the eigenvector problem.

The solutions $V_g$ have two important properties: they are only defined up to a phase,
and they are guaranteed to commute with the diagonal matrix $\Lambda$ of Schmidt weights.  

Due to the first property, these operators are not guaranteed to form a linear
representation of the group of on-site symmetries, but in general could make
up a projective representation, satisfying
$$V_g V_h = e^{i \omega(g, h)} V_{gh}.$$ 
It is not always possible to absorb these phases into the definitions of the $V_g$.
The set of equivalent classes of phases $\omega(g, h)$ 
under redefinitions $V_g \to \alpha(g)V_g$ is called $H^2(G, U(1))$, the second group 
cohomology with $U(1)$ coefficients. 

For all the groups discussed in this paper, the group cohomology classes are labeled by
elements of a discrete abelian group - these discrete classes cannot be connected to each other
continuously. Physically, only a phase transition or breaking the symmetry allows one to connect
the different projective symmetry actions. 
Additionally, the classification of projective representations for the on-site symmetry group 
$U(1) \times \mathbb{Z}_W$ representing charge and translation around the cylinder is trivial. 
Thus, these edge symmetries can be taken to act linearly, 
and all Schmidt states can always be simultaneously assigned charge and momentum eigenvalues,
as in Figure~\ref{fig:ESL910}.

The second property guarantees that the $V_g$ only mixes exactly degenerate Schmidt states.
The action of $V_g$ must have the same phases $\omega(g, h)$ on each degenerate block of Schmidt 
states, so the projective representation can be nontrivial on any block only if every Schmidt 
state throughout the entire spectrum is degenerate. The degeneracy will be protected by the 
symmetry if and only if the $V_g$ form a nontrivial projective representation. Therefore
this 1D SPT analysis can only potentially give a nontrival answer for the odd $W$ states 
of the HFBI, where this exact degeneracy is seen throughout the spectrum. Nonetheless, we find 
there are no on-site symmetries of the wavefunction that can be used to explain the degenerate
entanglement spectrum of the odd $W$ HFBI states. Instead we must use an inversion symmetry. 

The MPS analysis of inversion symmetry works similarly. We will consider in general any symmetry  
$h$ of the wavefunction that squares to the identity and that can be written in the MPS as the 
product of an on-site symmetry action $U_h$ and a transpose of the site tensor.
This will include an inversion of the honeycomb lattice - equivalent to a 180 degree rotation 
about the center of any plaquette, which we label $\I = \I_y \I_x$, and the combination of
inversion with on-site symmetries. In addition, by blocking two site-tensors together, we
can write the reflection symmetry $\I_y$ in this form as well. In this scenario, the 
edge symmetry action satisfies
\begin{center}
\beq
\label{eq:onsitesym}
\eeq
\vskip-5em
\includegraphics[width=0.6\columnwidth]{inv_sym.pdf}.
\end{center}
The map $V_{h}$ is also computed as a dominant eigenvector. 

For the HFBI, the symmetry group respected by the cylinder geometry is
$U(1) \times (\mathbb{Z}_W \rtimes \mathbb{Z}_2)
\times \mathbb{Z}_2^P \times \mathbb{Z}_2^T$, where the factors refer to charge
symmetry, translation around the cylinder, $\I_x$, $\I_y$, and $\tau$ respectively.
The $P$ and $T$ denote space-reversing and time-reversing symmetries, and signify
that they should be treated as antiunitary when computing the cohomology class
$H^2(U(1) \times (\mathbb{Z}_W \rtimes \mathbb{Z}_2)
\times \mathbb{Z}_2^P \times \mathbb{Z}_2^T ; U(1))$. Many of the non-trivial projective
representations of such a complicated group will remain projective when the symmetry is
restricted to a subgroup - in this case, the full symmetry is not needed to protect the 
entanglement degeneracy. As shown in Table~\ref{table:sym}, the projective representation 
corresponding to the HFBI state can indeed be protected by any one of a number of subgroups 
of the full symmetry group, all involving inversions and charge parity.

The symmetry actions - both on-site and inversion symmetries -
are computed in the Schmidt basis, but can be transformed
into the basis $\vket{\{\sigma_i\}}$ determined by the virtual legs of the PEPS in
Figure~\ref{fig:FBI_PEPS_2}. 
Since
$$
V_{\mathcal{\pi} \mathcal{I}} V_{\mathcal{\pi} \mathcal{I}}^* = -I,
$$

the representation is in the nontrivial class of

$$
H^2(\mathbb{Z}_2^P; U(1)) = \mathbb{Z}_2.
$$


\section{Mode expansion and symmetry action of free-boson CFT}
\label{Appendix:CFT}

\section{Variants on the HFBI wavefunction}
\label{Appendix:Variants}

\appendix

\section{Determining the edge action of the symmetry using MPS}
\label{Appendix:MPS}

We can use the formalism of MPS to assign an action
of the symmetry on the Schmidt states, and in particular for the charge and 
translation on-site symmetries the Schmidt states can be simultaneously assigned
charge and translation quantum numbers. This method of discovering the symmetry 
action will reproduce the action discussed in \ref{sec:ES} and quantum numbers used
in the spectra plots shown in Fig.~\ref{fig:ESL910}.
  
Let's discuss this formalism briefly. In addition,
we will discuss a generalization of this method that allows us to numerically
determine the symmetry action of inversion symmetry on the Schmidt states,
as in Section~\ref{sec:symmetry}. 
Both of these discussions follow Ref.~\onlinecite{pollmann2010}.

These discussions start by finding tensors $\Gamma, \Lambda$ representing the 
so-called canonical form of the MPS, as
originally detailed in \cite{perezgarcia2008}. This canonical form provides the Schmidt
decomposition at each site in the lattice.
\beq
\ket{\psi} = \sum\limits_{\{p_i\}} \ldots \Lambda \Gamma_{p_0} \Lambda \Gamma_{p_1} \Lambda \Gamma_{p_2} \Lambda \ldots \ket{... p_0 p_1 p_2 ...}.
\eeq
As a reminder, each physical leg represents all $2W$ physical sites on a cylinder slice,
and each virtual leg represents all virtual indices that connect cylinder slices.
The change of basis to canonical form generally mixes the Hilbert spaces from these
virtual legs, so the resulting basis won't be local around the circumference
of the cylinder.

Each on-site symmetry of the wavefunction $U_g = \otimes_i u^i_g$, with $U_g
\ket{\psi} = e^{i \Theta_g} \ket{\psi}$ is assigned an operator $V_g$ that
acts on the virtual leg of the MPS, that satisfies the equation
\begin{center}
\beq
\label{eq:onsitesym}
\eeq
\vskip-5em
\includegraphics[width=0.6\columnwidth]{group_sym.pdf}.
\end{center}
This equation can be rewritten and solved as an eigenvector problem;
for an MPS with a nondegenerate largest transfer matrix eigenvalue,
this equation is guaranteed have a unique solution where the eigenvalue $e^{i \theta_g}$
is the largest eigenvalue of the eigenvector problem.

The solutions $V_g$ have two important properties: they are only defined up to a phase,
and they are guaranteed to commute with the diagonal matrix $\Lambda$ of Schmidt weights.  

Due to the first property, these operators are not guaranteed to form a linear
representation of the group of on-site symmetries, but in general could make
up a projective representation, satisfying
$$V_g V_h = e^{i \omega(g, h)} V_{gh}.$$ 
It is not always possible to absorb these phases into the definitions of the $V_g$.
The set of equivalent classes of phases $\omega(g, h)$ 
under redefinitions $V_g \to \alpha(g)V_g$ is called $H^2(G, U(1))$, the second group 
cohomology with $U(1)$ coefficients. 

For all the groups discussed in this paper, the group cohomology classes are labeled by
elements of a discrete abelian group - these discrete classes cannot be connected to each other
continuously. Physically, only a phase transition or breaking the symmetry allows one to connect
the different projective symmetry actions. 
Additionally, the classification of projective representations for the on-site symmetry group 
$U(1) \times \mathbb{Z}_W$ representing charge and translation around the cylinder is trivial. 
Thus, these edge symmetries can be taken to act linearly, 
and all Schmidt states can always be simultaneously assigned charge and momentum eigenvalues,
as in Figure~\ref{fig:ESL910}.

The second property guarantees that the $V_g$ only mixes exactly degenerate Schmidt states.
The action of $V_g$ must have the same phases $\omega(g, h)$ on each degenerate block of Schmidt 
states, so the projective representation can be nontrivial on any block only if every Schmidt 
state throughout the entire spectrum is degenerate. The degeneracy will be protected by the 
symmetry if and only if the $V_g$ form a nontrivial projective representation. Therefore
this 1D SPT analysis can only potentially give a nontrival answer for the odd $W$ states 
of the HFBI, where this exact degeneracy is seen throughout the spectrum. Nonetheless, we find 
there are no on-site symmetries of the wavefunction that can be used to explain the degenerate
entanglement spectrum of the odd $W$ HFBI states. Instead we must use an inversion symmetry. 

The MPS analysis of inversion symmetry works similarly. We will consider in general any symmetry  
$h$ of the wavefunction that squares to the identity and that can be written in the MPS as the 
product of an on-site symmetry action $U_h$ and a transpose of the site tensor.
This will include an inversion of the honeycomb lattice - equivalent to a 180 degree rotation 
about the center of any plaquette, which we label $\I = \I_y \I_x$, and the combination of
inversion with on-site symmetries. In addition, by blocking two site-tensors together, we
can write the reflection symmetry $\I_y$ in this form as well. In this scenario, the 
edge symmetry action satisfies
\begin{center}
\beq
\label{eq:invsym}
\eeq
\vskip-5em
\includegraphics[width=0.6\columnwidth]{inv_sym.pdf}.
\end{center}
The map $V_{h}$ is also computed as a dominant eigenvector. 

For the HFBI, the symmetry group respected by the cylinder geometry is
$U(1) \times (\mathbb{Z}_W \rtimes \mathbb{Z}_2)
\times \mathbb{Z}_2^P \times \mathbb{Z}_2^T$, where the factors refer to charge
symmetry, translation around the cylinder, $\I_x$, $\I_y$, and $\tau$ respectively.
The $P$ and $T$ denote space-reversing and time-reversing symmetries, and signify the 
antiunitary action on the Schmidt states. 
%Compute the cohomology class of 
%$H^2(U(1) \times (\mathbb{Z}_W \rtimes \mathbb{Z}_2)
%\times \mathbb{Z}_2^P \times \mathbb{Z}_2^T ; U(1)) ?$
Many of the non-trivial projective
representations of such a complicated group will remain projective when the symmetry is
restricted to a subgroup - in this case, the full symmetry is not needed to protect the 
entanglement degeneracy. As shown in Table~\ref{table:sym}, the projective representation 
corresponding to the HFBI state can indeed be protected by any one of a number of subgroups 
of the full symmetry group, all involving inversions and charge parity.

The symmetry actions - both on-site and inversion symmetries -
are computed in the Schmidt basis, but can be transformed
into the basis $\vket{\{\sigma_i\}}$ determined by the virtual legs of the PEPS in
Figure~\ref{fig:FBI_PEPS_2}. 
In this case, the symmetry action $V_{\I_y}$ is precisely
a particle-hole symmetry in the local PEPS basis, with coefficients
$$
V_{\I_y}\vket{\sigma_1, \ldots, \sigma_{W}} = \vket{1-\sigma_1, \ldots, 1-\sigma_{W}}, 
$$
since a state where the $i^{th}$ hexagon contributes $\sigma_i$ bosons on the right is paired with
a state where the $i^{th}$ hexagon contributes $1-\sigma_i$ on the left.
Thus 
$$
V_{\I_y} = \prod\limits_i \sigma_i^x K,
$$
where K is complex conjugation in the local PEPS basis, and $\sigma_i^x$ is the Pauli
operator acting on the $i^{th}$ site of the local PEPS basis.

Charge symmetry acts locally as well:
$$
e^{i \theta \mathcal{Q}}\vket{\sigma_1, \ldots, \sigma_{W}} = e^{i \theta \sum(\sigma_i-1/2)}\vket{\sigma_1, \ldots, \sigma_{W}}.
$$
In particular, charge parity $V_{\varPi} = e^{i \theta \mathcal{Q}}$ can be written as
$$V_{\varPi} = e^{i \pi \sum(\sigma_i - 1/2)} = \prod\limits_i \sigma_i^z.$$

The combined action of charge parity and reflection across the cut takes the form 
$$
V_{\varPi \I_y} = \prod\limits_i \left(i \sigma_i^y \right) K,
$$
which is precisely the form that time-reversal acting on an ordinary spin-$\frac12$ chain takes.
When the circumference of the cylinder $W$ is odd, we see that
$$V_{\varPi \I_y}V_{\varPi \I_y}^{*} = -I.$$ 
The degeneracy of the entanglement spectrum can be seen as an application of Kramer's theorem.
Formally, this property is said to characterize the nontrivial projective representation
$$
H^2(\mathbb{Z}_2^P; U(1)) = \mathbb{Z}_2,
$$
and remains true while $\varPi \I_y$ is a symmetry and no phase transitions have occurred.

Time reversal symmetry acts as complex conjugation in the local PEPS basis $V_{\tau}=K$.
Translation and $\I_x$ act as permutations of the local PEPS basis:
\begin{equation*}
\begin{split}
V_{T}\vket{\sigma_1, \ldots, \sigma_{W}} &= \vket{\sigma_2, \ldots, \sigma_{W}, \sigma_{1}} \\
V_{\I_x}\vket{\sigma_1, \ldots, \sigma_{W}} &= \vket{\sigma_W, \ldots, \sigma_{1}}.
\end{split}
\end{equation*}
These symmetries can be combined with $V_{\varPi \I_y}$ to create the additional topological 
invariants shown in Table~\ref{table:sym}. A non-trivial projective 
representation in $$H^2(\mathbb{Z}_2 \times \mathbb{Z}_2; U(1)) = \mathbb{Z}_2$$
is created whenever two unitary symmetries that commute in the bulk satisfy 
$$V_{g_1} V_{g_2} V_{g_1}^{-1} V_{g_2}^{-1} = -I.$$
Each new invariant is related to a new set of pertubations that can't break the entanglement 
degeneracy. 

\section{Local Hamiltonian for the Edge free boson CFT}
\label{Appendix:LocalEdge}

B

\section{Mode expansion and symmetry action of free-boson CFT}
\label{Appendix:CFT}
How do the symmetry protecting operations act on the CFT states
$\ket{e, m, \{n_i\}, \{\bar{n}_i\}}$?

From that, infer how they act on the free boson fields $\phi$ and $\theta$.

Which perturbations to the CFT gap it out completely or spontaneously break the symmetry?

Are those pertubations forbidden by the implementation of symmetry on the edge.

\section{Variants on the HFBI wavefunction}
\subsection{Tuning soft-core bosons to hard-core}
In Equations~\eqref{eqn:Dsc} and \eqref{eqn:Dhc}, the tensor $D$ can be replaced by a more general form 
\begin{equation} \label{eqn:Dgen}
D_{p, i_0 i_1 i_2}  = \left\{ \begin{array}{ll}
													d_p  &: p =i_0+i_1+i_2  \\
													0  &:  \text{else}
													\end{array}
											\right. ,
\end{equation}
which the coefficients $d_p = 1,\, 1,\, \sqrt{2},\,\sqrt{6}$ for $p = 0,\, 1,\, 2,\, 3$ in the soft-core state and $d_p = 1,\, 1,\, 0,\, 0$ for $p = 0,\, 1,\, 2,\, 3$ in the hard-core state. We can continously tune the coefficients $d_2$ and $d_3$ from the soft-core to the hard-core values.
Upon doing so, we find that the transfer matrix spectrum remains gapped, with the correlation length monotonically increasing from the soft-core state to the hard-core state. 
Furthermore, the low energy parts of the entanglement spectrum do not change significantly through this tuning.
Therefore we expect that the hard-core and soft-core phases can be adiabatically connected with a 
path of local Hamiltonians, and all SPT results that apply to one state apply to the other. By choosing appropriate values of $d_2$ and $d_3$, we can also make replace the vacuum $\ket{0}$ in
Equation~\eqref{eq:def} with a constant background of $N$ filled bosons $\ket{N}$, or even make
states of spins $S$ where Equation~\eqref{eq:def} becomes
\begin{equation} \label{eq:spindef}
\ket{\psi} = \prod\limits_{\varhexagon} \left( \sum\limits_{i \in \varhexagon} S^{+}_{i} \right) \ket{S_z = m}.
\end{equation}

\subsection{Inversion Protected Phase}
Additionally, the tensor $W$ in Equation~\eqref{eq:W} can be replaced by the more general form
\begin{equation} \label{eq:Wgen}
W^{n_1 \ldots n_6}  = \left\{ \begin{array}{lr}
													p_x  : & n_x=1,\, n_y = 0
													\; \forall \; y \neq x \\
													0  : & \text{else}
													\end{array} \right.,
\end{equation}
which corresponds to modifying Equation~\eqref{eq:def} to 
\begin{equation} \label{eq:pdef}
\ket{\psi_{\ell}} = \prod\limits_{\varhexagon} \left( \sum\limits_{i \in \varhexagon} p_i b^{\dagger}_{i} \right) \ket{0}.
\end{equation}

This does not in general preserve the rotational symmetry of the state, but it does if the 
coefficients $p_0, \ldots p_5$ are in an angular momentum mode $$p_x = e^{i x \ell}$$ where
$\ell \in \{0, 2\pi/6, \ldots 5\pi/6 \}$. These 6 discrete solutions can't be continously tuned to one another while preserving all the lattice symmetries.

The state $\ket{\psi_{\ell=\pi}}$ can be shown to be related to state $\ket{\psi_{\ell=0}}$ discussed in the main text by a on-site unitary operator $U(\pi)$, 
where 
\begin{equation} \label{eq:Uphi}
U(\varphi) = \prod\limits_{j \in B} e^{i \varphi \hat{Q}_j}.
\end{equation}
Due to this relation, $\ket{\psi_{\ell=\pi}}$ and $\ket{\psi_{\ell=0}}$ have identical correlation lengths and entanglement spectra. 
However, the protecting symmetries from Table~\ref{table:sym} are mapped using conjugation
by $U(\pi)$ into a new set of protecting symmetries, shown in Table~\ref{table:pisym}. 

\begin{table}
\begin{tabular*}{\columnwidth}{@{\extracolsep{\stretch{1}}}*{4}{r}@{}}
\toprule
Group & Generators & Invariant & $i$  \\
\midrule
$\mathbb{Z}_2^P$ & $\{\I \}$ 
& $V_{\I} V_{\I}^* = -I$ &$-$ \\
$\mathbb{Z}_2^P$ & $\{\I_y \}$ 
&$V_{\I_y} V_{\I_y}^* = -I$ &$-$ \\ \hline
$\mathbb{Z}_2 \times \mathbb{Z}_2^{PT}$& $\{\varPi, \tau \varPi \I\}$ 
&$V_{\varPi} V_{\tau \varPi \I} V_{\varPi}^{-1} V_{\tau \varPi \I}^{-1} = -I$ &$+$ \\
$\mathbb{Z}_2 \times \mathbb{Z}_2^{PT}$& $\{\varPi, \tau \varPi \I_y\}$
&$V_{\varPi} V_{\tau \varPi \I_y} V_{\varPi}^{-1} V_{\tau \varPi \I_y}^{-1} = -I$ &$+$ \\
$\mathbb{Z}_2 \times \mathbb{Z}_2^{PT}$& $\{\varPi \I_x, \tau \varPi \I\}$
&$V_{\varPi \I_x} V_{\tau \varPi \I} V_{\varPi \I_x}^{-1} V_{\tau \varPi \I}^{-1} = -I$&$+$ \\
$\mathbb{Z}_2 \times \mathbb{Z}_2^{PT}$& $\{\varPi \I_x, \tau \varPi \I_y\}$
&$V_{\varPi \I_x} V_{\tau \varPi \I_y} V_{\varPi \I_x}^{-1} V_{\tau \varPi \I_y}^{-1} = -I$ &$+$\\
\bottomrule
\end{tabular*}
\caption{Summary of symmetry protecting invariants found for the $\ket{\psi_{\ell=\pi}}$ state. 
The degenerate entanglement spectrum cannot be split unless all 6 of the  minimal protecting symmetry groups are broken. }
\label{table:pisym}
\end{table}

Notably,
since
$$
U(\pi) \varPi \I U(\pi)^{\dagger} = \I, 
$$
this state has doubly degenerate entanglement spectra on odd cylinder sizes protected by lattice inversion symmetry alone.

A similar mapping for 1-D inversion protected states is discussed in Appendix A of Ref.~\onlinecite{pollmann2010}. As discussed \brayden{somewhere}, the state $\ket{\psi_{\ell=0}}$
on the $W=1$ cylinder is adiabatically connected to the 1-D Haldane insulator state discussed in
Ref.~\onlinecite{pollmann2010}. The new $\ket{\psi_{\ell=\pi}}$ state on the $W=1$ cylinder is instead adiabatically connected to the 1-D AKLT state. 

\label{Appendix:Variants}

\acknowledgements
B.W. would like to acknowledge Eugeniu Plamadeala for helpful discussions.

\bibliography{fbi}

\end{document}
