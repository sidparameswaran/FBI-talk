% y=\sqrt{3/4}*(minimum size)/2
%

%http://tex.stackexchange.com/questions/6019/drawing-hexagons
\begin{tikzpicture}[x=7.5mm,y=4.33mm]

\clip (-2, -2.4) rectangle (4, 3.6);

  \tikzset{box/.style={
      regular polygon,
      regular polygon sides=6,
      minimum size=10mm,
      inner sep=0mm,
      outer sep=0mm,
      rotate=0,
      dotted,
      thin,
      black,
      fill=white,
      draw}}
 \tikzset{wb/.style={
 	regular polygon,
	regular polygon sides=6,
	minimum size=3mm,
      inner sep=0mm,
      outer sep=0mm,
      rotate=0,
      % dotted,
      very thick,
      blue,
      fill=blue!30,
      draw}}
      
\tikzset{operator/.style={
	circle=1pt,
	draw=orange!100, 
	very thick,
	fill=orange!80,
	inner sep=2pt}}


\foreach \i in {-2,...,2}{
	\foreach \j in {-2, ...,  2} {
		\pgfmathtruncatemacro{\x}{2*\i+1}
		\pgfmathtruncatemacro{\y}{2*\j+1}
             \node[box] (H\x\y) at (\x,\y) {};
             \node[wb] (W\x\y) at (\x,\y) {};
           \foreach \k in {1,...,6}{
              	\node[operator] (D\x\y\k) at (H\x\y.corner \k) {};
              	\draw[ thick, black] (W\x\y) -- (D\x\y\k);
              	 \node[black] at (H\x\y.corner \k){\tiny D};
              	}
             \node[black] at (\x, \y) {\tiny W};
        }
}

\foreach \i in {-2,...,2}{
	\foreach \j in {-2,...,2} {
		\pgfmathtruncatemacro{\x}{2*\i}
		\pgfmathtruncatemacro{\y}{2*\j}
             \node[box] (H\x\y) at (\x,\y) {};
             \node[wb] (W\x\y) at (\x,\y) {};
             \node[black] at (\x, \y) {\tiny W};
              \foreach \k in {1,...,6}{
              	\node[operator] (D\x\y\k) at (H\x\y.corner \k) {};
              	\draw[ thick, black] (W\x\y) -- (D\x\y\k);
              	\node[black] at (H\x\y.corner \k){\tiny D};
             	}
        }
}

%\draw[thick, black] (-2, -2.4) rectangle (2, 3.6);

\end{tikzpicture}